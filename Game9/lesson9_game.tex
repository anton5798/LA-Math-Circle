\documentclass[a4paper,12pt]{article} 
\usepackage{amsmath}
\usepackage[retainorgcmds]{IEEEtrantools}
\usepackage{graphicx}
\usepackage{amsfonts}
\usepackage{amsthm}
%\usepackage{centernot}
%\usepackage{setspace}
\usepackage{mathtools}
\usepackage{amssymb}
%\usepackage{bm}
%\usepackage[mathscr]{euscript}
%\usepackage{pictexwd,dcpic}
%\usepackage{tikz}
%\usepackage{tikz-cd}
\usepackage[margin=1in]{geometry}
\usepackage{breqn}
\newtheoremstyle{perfect}% name
  {}%         Space above, empty = `usual value'
  {}%         Space below
  {}% Body font
  {}%         Indent amount (empty = no indent, \parindent = para indent)
  {\bfseries}% Thm head font
  {.}%        Punctuation after thm head
  {\newline}% Space after thm head: \newline = linebreak
  {}%         Thm head spec
\theoremstyle{perfect}
\newtheorem{lem}{Lemma}
\newtheorem{thm}{Theorem}
\newtheorem{dfn}{Definition}
\newtheorem{exm}{Example}
\newtheorem{prop}{Proposition}
\newtheorem{crl}{Corollary}
\newtheorem{rem}{Reminder}
\newtheorem{prb}{Problem}
\newtheorem{exe}{Exercise}
\makeatletter
\newenvironment{cprb}[1]
  {\count@\c@prb
   \global\c@prb#1
    \global\advance\c@prb\m@ne
   \prb}
  {\endprb
   \global\c@prb\count@}
\makeatother

\makeatletter
\newenvironment{cexe}[1]
  {\count@\c@exe
   \global\c@exe#1
    \global\advance\c@exe\m@ne
   \exe}
  {\endexe
   \global\c@exe\count@}
\makeatother

\newcommand{\varline}{0}
\newcommand{\gen}[1]{\left< #1 \right>}
\newcommand{\ngen}[1]{\gen{\gen{#1}}}
\newcommand{\ov}[1]{\,\overline{#1}}

\renewcommand{\Re}{\re}
\renewcommand{\Im}{\imm}
\newcommand{\eps}{\varepsilon}
\newcommand{\Mor}{\text{Mor}}
\newcommand{\cir}[1]{\mathrlap{\bigcirc}{\;#1}}
\newcommand{\Z}{\mathbb{Z}}
\newcommand{\Q}{\mathbb{Q}}
\newcommand{\R}{\mathbb{R}}
\newcommand{\C}{\mathbb{C}}
\newcommand{\F}{\mathbb{F}}
\newcommand{\N}{\mathbb{N}}
\newcommand{\lnorm}{\vartriangleleft}
\newcommand{\rnorm}{\vartriangleright}
\newcommand{\id}{\text{id}}

\makeatletter
\renewcommand\part[1]{
\ifnum\pdfstrcmp{\varline}{1}=0
    \vspace{.10in}\textbf{\\#1)}
  \else
    \textbf{#1)}
  \fi\renewcommand{\varline}{1}}
\makeatother

\makeatletter
\newcommand{\tpmod}[1]{{\@displayfalse\pmod{#1}}}
\makeatother
\renewcommand{\restriction}{\mathord{\upharpoonright}}


\author{Anton Lykov \& Konstantin Miagkov} 
\title{Lesson 9: Game}
\begin{document} 
%\setstretch{1}
\setlength{\parindent}{0cm}
\maketitle

\begin{prb}
Magnus played in a chess tournament with 20 games, and earned 12.5 points.
How many more wins than losses did Magnus have? In a chess tournament, you earn 1 point for a win, 0.5 for a draw, and 0 for a loss. 
\end{prb} 

\begin{prb}
Amy makes two 4-digit numbers using each of the digits 1,2,3,4,5,6,7, and 8 exactly once. If Amy makes the numbers so that adding them gives the smallest possible total, what is the total?
\end{prb} 

\begin{prb}
	Mike has 130 details. He can build a toy windmill using 5 details, a ship using 7 details, and a plane using 14 details. A plane costs 19 coins, ship - 8 coins, and windmill - 6 coins. What is the largest amount of coins Mike can earn?
\end{prb} 

\begin{prb}
Find a 10-digit number where the first digit is equal to how many 0's are in the number, the second digit is equal to how many 1's are in the number, the third digit is equal to how many 2's are in the number, all the way up to the last digit, which is equal to how many 9's are in the number.
\end{prb}
 
\begin{prb}
Mark 6 points on the plane such that for each point there would be exactly three other points at a distance 1 from it.
\end{prb}

\begin{prb}
In the $\triangle ABC$ the median $BM$ is perpendicular to the angle bisector $AD$. If $AC = 12$, find $AB$.
\end{prb}

\begin{prb}
Find a point $(x,y)$ that satisfies both $y = x^3 + 3x^2 + 1$ and $y = x^3 + 2x^2 + 2x$.
\end{prb}

\begin{prb}
Find the smallest positive integer $n$ such that any positive integer $a$ has the same last digit as $a^n$.
\end{prb}

\begin{prb}
Let x and y denote the real roots of the equation $x^2 - 3^{2011}x + 3^{4020} = 0$. Find $$\log_3\left(\frac{x^3 + y^3}{2}\right)$$
\end{prb}

\begin{prb}
The IQs of Jon (21), John (22), and Johhn (23) follow a quadratic function\\ $f(x) = -45x^2 + bx + c$, where $x$ represents the age in years after 20 years old. So for example, John's IQ at 22 years old is $f(2)$. Given that John's IQ is 130 and Johhn's IQ is 100, what is Jon's IQ?
\end{prb}


\begin{prb}
Find all real values of $m$ for which the equations $mx-1000=1001$ and $1001x = m - 1000x$ have a common root.
\end{prb}

\begin{prb}
Put the following numbers in the increasing order: $2^{222}, 2^{22^2}, 222^2, 2^{2^{2^2}}, 2^{2^{22}}, 22^{22}, 22^{2^2}$. 
\end{prb}


\begin{prb}
What's the biggest number of non-overlapping $1\times 4$ tiles that can be placed in a $6\times 6$ square?
\end{prb}

\begin{prb}
Jon chose with three distinct nonzero digits. John added up all two-digit numbers using only those three digits, and his result was $231$. What digits did Jon choose?
\end{prb}

\begin{prb}
Some squares of the $100\times 100$ board have stones on them. For every stone on the board, either its row or its column contain exactly one stone. What is the maximal possible number of stones on the board?
\end{prb}

\begin{prb}
You know that $a+b+c=7$ and $$\frac{1}{a+b} + \frac{1}{b+c} + \frac{1}{a+c} = 0.7$$ Find $$\frac{c}{a+b} + \frac{a}{b+c} + \frac{b}{a+c}$$
\end{prb}

\begin{prb}
$$\frac{1^2 + 2^2 + ... + n^2}{1 + 2 + ... + n} = 19$$
What is $n$?
\end{prb}

\begin{prb}
How many unique tilings of a $2 \times 11$ grid are there using $1 \times 2$ and $2 \times 1$ tiles? (A tiling being completely filling up the grid with the tiles and having no extra slots).
\end{prb}

\begin{prb}
Compute the sum of all the roots of 
$$(2x+3)(x-4)+(2x+3)(x-6)=0$$
\end{prb}

\begin{prb}
Let $a + 1 = b + 2 = c + 3 = d + 4 = a + b + c + d + 5$. What is $a + b + c + d$?
\end{prb}


\end{document}