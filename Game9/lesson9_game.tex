\documentclass[a4paper,12pt]{article} 
\usepackage{amsmath}
\usepackage[retainorgcmds]{IEEEtrantools}
\usepackage{graphicx}
\usepackage{amsfonts}
\usepackage{amsthm}
%\usepackage{centernot}
%\usepackage{setspace}
\usepackage{mathtools}
\usepackage{amssymb}
%\usepackage{bm}
%\usepackage[mathscr]{euscript}
%\usepackage{pictexwd,dcpic}
%\usepackage{tikz}
%\usepackage{tikz-cd}
\usepackage[margin=1in]{geometry}
\usepackage{breqn}
\newtheoremstyle{perfect}% name
  {}%         Space above, empty = `usual value'
  {}%         Space below
  {}% Body font
  {}%         Indent amount (empty = no indent, \parindent = para indent)
  {\bfseries}% Thm head font
  {.}%        Punctuation after thm head
  {\newline}% Space after thm head: \newline = linebreak
  {}%         Thm head spec
\theoremstyle{perfect}
\newtheorem{lem}{Lemma}
\newtheorem{thm}{Theorem}
\newtheorem{dfn}{Definition}
\newtheorem{exm}{Example}
\newtheorem{prop}{Proposition}
\newtheorem{crl}{Corollary}
\newtheorem{rem}{Reminder}
\newtheorem{prb}{Problem}
\newtheorem{exe}{Exercise}
\makeatletter
\newenvironment{cprb}[1]
  {\count@\c@prb
   \global\c@prb#1
    \global\advance\c@prb\m@ne
   \prb}
  {\endprb
   \global\c@prb\count@}
\makeatother

\makeatletter
\newenvironment{cexe}[1]
  {\count@\c@exe
   \global\c@exe#1
    \global\advance\c@exe\m@ne
   \exe}
  {\endexe
   \global\c@exe\count@}
\makeatother

\newcommand{\varline}{0}
\newcommand{\gen}[1]{\left< #1 \right>}
\newcommand{\ngen}[1]{\gen{\gen{#1}}}
\newcommand{\ov}[1]{\,\overline{#1}}

\renewcommand{\Re}{\re}
\renewcommand{\Im}{\imm}
\newcommand{\eps}{\varepsilon}
\newcommand{\Mor}{\text{Mor}}
\newcommand{\cir}[1]{\mathrlap{\bigcirc}{\;#1}}
\newcommand{\Z}{\mathbb{Z}}
\newcommand{\Q}{\mathbb{Q}}
\newcommand{\R}{\mathbb{R}}
\newcommand{\C}{\mathbb{C}}
\newcommand{\F}{\mathbb{F}}
\newcommand{\N}{\mathbb{N}}
\newcommand{\lnorm}{\vartriangleleft}
\newcommand{\rnorm}{\vartriangleright}
\newcommand{\id}{\text{id}}

\makeatletter
\renewcommand\part[1]{
\ifnum\pdfstrcmp{\varline}{1}=0
    \vspace{.10in}\textbf{\\#1)}
  \else
    \textbf{#1)}
  \fi\renewcommand{\varline}{1}}
\makeatother

\makeatletter
\newcommand{\tpmod}[1]{{\@displayfalse\pmod{#1}}}
\makeatother
\renewcommand{\restriction}{\mathord{\upharpoonright}}


\author{Anton Lykov \& Konstantin Miagkov} 
\title{Lesson 9: Game}
\begin{document} 
%\setstretch{1}
\maketitle

\begin{prb}
Magnus played in a chess tournament with 20 games, and earned 12.5 points.
How many more games did he win, than lose? In a chess tournament, you earn 1 point for a win, 0.5 for a draw, and 0 for a lose. 
\end{prb} 

\begin{prb}
Amy makes two 4-digit numbers using each of the digits 1,2,3,4,5,6,7, and 8 exactly once. If Amy makes the numbers so that adding them gives the smallest possible total, what is the total?
\end{prb} 

\begin{prb}
	Mike has 130 details. He can build a toy windmill using 5 details, a ship using 7 details, and a plane using 14 details. A plane costs 19 coins, ship - 8 coins, and windmill - 6 coins. What is the largest amount of coins Mike can earn?
\end{prb} 

\begin{prb}
Find a 10-digit number where the first digit is equal to how many 0's are in the number, the second digit is equal to how many 1's are in the number, the third digit is equal to how many 2's are in the number, all the way up to the last digit, which is equal to how many 9's are in the number.
\end{prb}
 

\begin{prb}
Mark 6 points on the plane such that for each point there would be exactly three other points at a distance 1 from it.
\end{prb}




\end{document}