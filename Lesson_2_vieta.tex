\documentclass[a4paper,12pt]{article} 
\usepackage{amsmath}
\usepackage[retainorgcmds]{IEEEtrantools}
\usepackage{graphicx}
\usepackage{amsfonts}
\usepackage{amsthm}
\usepackage{centernot}
\usepackage{setspace}
\usepackage{mathtools}
\usepackage{amssymb}
\usepackage{bm}
\usepackage[mathscr]{euscript}
\usepackage{pictexwd,dcpic}
\usepackage{tikz}
\usepackage{tikz-cd}
%\usepackage[margin=1in]{geometry}
\usepackage{breqn}
\newtheoremstyle{perfect}% name
  {}%         Space above, empty = `usual value'
  {}%         Space below
  {}% Body font
  {}%         Indent amount (empty = no indent, \parindent = para indent)
  {\bfseries}% Thm head font
  {.}%        Punctuation after thm head
  {\newline}% Space after thm head: \newline = linebreak
  {}%         Thm head spec
\theoremstyle{perfect}
\newtheorem{lem}{Lemma}
\newtheorem{thm}{Theorem}
\newtheorem{dfn}{Definition}
\newtheorem{exm}{Example}
\newtheorem{prop}{Proposition}
\newtheorem{crl}{Corollary}
\newtheorem{rem}{Reminder}
\newtheorem{prb}{Problem}
\newtheorem{exe}{Exercise}
\makeatletter
\newenvironment{cprb}[1]
  {\count@\c@prb
   \global\c@prb#1
    \global\advance\c@prb\m@ne
   \prb}
  {\endprb
   \global\c@prb\count@}
\makeatother
\makeatletter
\newenvironment{cexe}[1]
  {\count@\c@exe
   \global\c@exe#1
    \global\advance\c@exe\m@ne
   \exe}
  {\endexe
   \global\c@exe\count@}
\makeatother
\newcommand{\varline}{0}
\newcommand{\gen}[1]{\left< #1 \right>}
\newcommand{\ngen}[1]{\gen{\gen{#1}}}
\newcommand{\ov}[1]{\,\overline{#1}}
\DeclareMathOperator{\tor}{Tor}
\DeclareMathOperator{\aut}{Aut}
\DeclareMathOperator{\inn}{Inn}
\DeclareMathOperator{\im}{im}
\DeclareMathOperator{\imm}{Im}
\DeclareMathOperator{\ad}{ad}
\DeclareMathOperator{\Ad}{Ad}
\DeclareMathOperator{\Sp}{Sp}
\DeclareMathOperator{\SO}{SO}
\DeclareMathOperator{\SL}{SL}
\DeclareMathOperator{\SU}{SU}
\DeclareMathOperator{\GL}{GL}
\DeclareMathOperator{\PGL}{PGL}
\DeclareMathOperator{\re}{Re}
\DeclareMathOperator{\Hom}{Hom}
\DeclareMathOperator{\sym}{Sym}
\DeclareMathOperator{\ind}{Ind}
\DeclareMathOperator{\res}{Res}
\DeclareMathOperator{\sgn}{sgn}
\DeclareMathOperator{\End}{End}
\DeclareMathOperator{\colim}{colim}
\DeclareMathOperator{\coker}{coker}
\DeclareMathOperator{\Tr}{Tr}
\DeclareMathOperator{\intr}{int}
\DeclareMathOperator{\extr}{ext}
\DeclareMathOperator{\chr}{char}
\DeclareMathOperator{\supp}{supp}
\DeclareMathOperator{\hol}{Hol}
\DeclareMathOperator{\spec}{Spec}
\renewcommand{\Re}{\re}
\renewcommand{\Im}{\imm}
\newcommand{\eps}{\varepsilon}
\newcommand{\Mor}{\text{Mor}}
\newcommand{\cir}[1]{\mathrlap{\bigcirc}{\;#1}}
\newcommand{\Z}{\mathbb{Z}}
\newcommand{\Q}{\mathbb{Q}}
\newcommand{\R}{\mathbb{R}}
\newcommand{\C}{\mathbb{C}}
\newcommand{\F}{\mathbb{F}}
\newcommand{\N}{\mathbb{N}}
\newcommand{\lnorm}{\vartriangleleft}
\newcommand{\rnorm}{\vartriangleright}
\newcommand{\id}{\text{id}}
\newcommand{\dd}[1]{\mathrm{d}{#1}}
\newcommand{\p}[1]{\left( #1 \right)}
\newcommand{\parder}[2]{\frac{\partial #1}{\partial #2}}
\newcommand{\legendre}[2]{\left(\frac{#1}{#2}\right)}
\makeatletter
\renewcommand\part[1]{
\ifnum\pdfstrcmp{\varline}{1}=0
    \vspace{.10in}\textbf{\\#1)}
  \else
    \textbf{#1)}
  \fi\renewcommand{\varline}{1}}
\makeatother
\makeatletter
\newcommand{\tpmod}[1]{{\@displayfalse\pmod{#1}}}
\makeatother
\renewcommand{\restriction}{\mathord{\upharpoonright}}\author{Konstantin Miagkov} 
\title{Lesson 2: Vieta's formula}
\begin{document} 
%\setstretch{1}
\maketitle

\begin{dfn}
We say that $x_0$ is a root of a function $f(x)$ if $f(x_0) = 0$.
\end{dfn}

\begin{prb}
\part{a} Let $ax^2+bx+c = 0$ be a quadratic equation. Show that if it has two distinct real roots $x_0, x_1$, then $ax^2+bx+c = a(x - x_0)(x - x_1)$. Hint: consider the difference between $ax^2+bx+c$ and $a(x - x_0)(x - x_1)$. What degree is it? How many roots does it have?
\part{b} Show that a quadratic equation cannot have more than two distinct real roots.
\part{c} Now suppose that $ax^2+bx+c$ has exactly one real root $x_0$. Show that $ax^2+bx+c = a(x-x_0)^2$.
\end{prb}

\begin{prb}
\part{a}[Vieta's formulas] Consider a quadratic equation $ax^2+bx+c = 0$ with real roots $x_0, x_1$. Show that $$x_0 + x_1 = -b/a$$ $$x_0x_1 = c/a$$ These are called \textit{Vieta's formulas}. You may use the result of problem 1 even if you did not solve it.
\part{b} Given any two real numbers $x_0, x_1$ with $x_0 + x_1 = u$ and $x_0x_1 = v$, show that both $x_0$ and $x_1$ are roots of the quadratic equation $x^2 - ux+v = 0$.
\end{prb}

\begin{prb}
\part{a} Let $x_0, x_1$ be roots of a quadratic equation $ax^2+bx+c = 0$.  Find the formula for $x_0^2 + x_1^2$ in terms of $a,b,c$.
\part{b} Let $x_0, x_1$ be roots of $x^2+bx+c = 0$.  Find the formula for $x_0^3 + x_1^3$ in terms of $b,c$.
\end{prb}

\begin{prb}
Consider a circle whose diameter is the side $AB$ of the triangle $ABC$. Show that if that circle contains the midpoint of $AC$, then $\triangle ABC$ is isosceles.
\end{prb}

\begin{prb}
Let $AC$ be a diameter of a circle, and $B$ be a point on the circle distinct from $A$ and $C$. Let $P$ be the foot of the perpendicular from $A$ to the tangent to the circle at $B$. Show that $AB$ is the angle bisector of $\angle PAC$. 
\end{prb}









\end{document}