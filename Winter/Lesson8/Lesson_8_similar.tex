\documentclass[a4paper,12pt]{article} 
\usepackage{amsmath}
\usepackage[retainorgcmds]{IEEEtrantools}
\usepackage{graphicx}
\usepackage{amsfonts}
\usepackage{amsthm}
\usepackage{centernot}
\usepackage{setspace}
\usepackage{mathtools}
\usepackage{amssymb}
\usepackage{bm}
\usepackage[mathscr]{euscript}
\usepackage{pictexwd,dcpic}
\usepackage{tikz}
\usepackage{tikz-cd}
\usepackage[margin=1in]{geometry}
\usepackage{breqn}
\newtheoremstyle{perfect}% name
  {}%         Space above, empty = `usual value'
  {}%         Space below
  {}% Body font
  {}%         Indent amount (empty = no indent, \parindent = para indent)
  {\bfseries}% Thm head font
  {.}%        Punctuation after thm head
  {\newline}% Space after thm head: \newline = linebreak
  {}%         Thm head spec
\theoremstyle{perfect}
\newtheorem{lem}{Lemma}
\newtheorem{thm}{Theorem}
\newtheorem{dfn}{Definition}
\newtheorem{exm}{Example}
\newtheorem{prop}{Proposition}
\newtheorem{crl}{Corollary}
\newtheorem{rem}{Reminder}
\newtheorem{prb}{Problem}
\newtheorem{exe}{Exercise}
\makeatletter
\newenvironment{cprb}[1]
  {\count@\c@prb
   \global\c@prb#1
    \global\advance\c@prb\m@ne
   \prb}
  {\endprb
   \global\c@prb\count@}
\makeatother
\makeatletter
\newenvironment{cexe}[1]
  {\count@\c@exe
   \global\c@exe#1
    \global\advance\c@exe\m@ne
   \exe}
  {\endexe
   \global\c@exe\count@}
\makeatother
\newcommand{\varline}{0}
\newcommand{\gen}[1]{\left< #1 \right>}
\newcommand{\ngen}[1]{\gen{\gen{#1}}}
\newcommand{\ov}[1]{\,\overline{#1}}
\DeclareMathOperator{\tor}{Tor}
\DeclareMathOperator{\aut}{Aut}
\DeclareMathOperator{\inn}{Inn}
\DeclareMathOperator{\im}{im}
\DeclareMathOperator{\imm}{Im}
\DeclareMathOperator{\ad}{ad}
\DeclareMathOperator{\Ad}{Ad}
\DeclareMathOperator{\Sp}{Sp}
\DeclareMathOperator{\SO}{SO}
\DeclareMathOperator{\SL}{SL}
\DeclareMathOperator{\SU}{SU}
\DeclareMathOperator{\GL}{GL}
\DeclareMathOperator{\PGL}{PGL}
\DeclareMathOperator{\re}{Re}
\DeclareMathOperator{\Hom}{Hom}
\DeclareMathOperator{\sym}{Sym}
\DeclareMathOperator{\ind}{Ind}
\DeclareMathOperator{\res}{Res}
\DeclareMathOperator{\sgn}{sgn}
\DeclareMathOperator{\End}{End}
\DeclareMathOperator{\colim}{colim}
\DeclareMathOperator{\coker}{coker}
\DeclareMathOperator{\Tr}{Tr}
\DeclareMathOperator{\intr}{int}
\DeclareMathOperator{\extr}{ext}
\DeclareMathOperator{\chr}{char}
\DeclareMathOperator{\supp}{supp}
\DeclareMathOperator{\hol}{Hol}
\DeclareMathOperator{\spec}{Spec}
\renewcommand{\Re}{\re}
\renewcommand{\Im}{\imm}
\newcommand{\eps}{\varepsilon}
\newcommand{\Mor}{\text{Mor}}
\newcommand{\cir}[1]{\mathrlap{\bigcirc}{\;#1}}
\newcommand{\Z}{\mathbb{Z}}
\newcommand{\Q}{\mathbb{Q}}
\newcommand{\R}{\mathbb{R}}
\newcommand{\C}{\mathbb{C}}
\newcommand{\F}{\mathbb{F}}
\newcommand{\N}{\mathbb{N}}
\newcommand{\lnorm}{\vartriangleleft}
\newcommand{\rnorm}{\vartriangleright}
\newcommand{\id}{\text{id}}
\newcommand{\dd}[1]{\mathrm{d}{#1}}
\newcommand{\p}[1]{\left( #1 \right)}
\newcommand{\parder}[2]{\frac{\partial #1}{\partial #2}}
\newcommand{\legendre}[2]{\left(\frac{#1}{#2}\right)}
\makeatletter
\renewcommand\part[1]{
\ifnum\pdfstrcmp{\varline}{1}=0
    \vspace{.10in}\textbf{\\#1)}
  \else
    \textbf{#1)}
  \fi\renewcommand{\varline}{1}}
\makeatother
\makeatletter
\newcommand{\tpmod}[1]{{\@displayfalse\pmod{#1}}}
\makeatother
\renewcommand{\restriction}{\mathord{\upharpoonright}}\author{Konstantin Miagkov} 
\title{Lesson 8: Similar Triangles}
\begin{document} 
%\setstretch{1}
\maketitle

\begin{dfn}
Two triangles are called \textit{similar} is they have the same angles.  
\end{dfn}

\begin{thm}
Triangles $\triangle ABC$ and $\triangle A'B'C'$ are similar if and only if $$\frac{AB}{A'B'} = \frac{BC}{B'C'} = \frac{AC}{A'C'}$$
In other words, they have the same ratio of sides. This common ration is called the \textit{similarity ratio}.
\end{thm}
\begin{thm}
Triangles $\triangle ABC$ and $\triangle A'B'C'$ are similar if and only if $\angle BAC = \angle B'A'C'$ and $$\frac{AB}{A'B'} = \frac{AC}{A'C'}$$
In other words, they share an angle and have the same ratio of sides adjacent to that angle.
\end{thm}
\begin{prb}
In lecture we proved theorem 1. Now you need to prove theorem 2. Here are the suggested steps. If the triangles are already similar, you may use the already proven theorem 1 or the intercept theorem to get the condition on the ratios of sides. The more difficult direction is to show that if $\angle BAC = \angle B'A'C'$ and $$\frac{AB}{A'B'} = \frac{AC}{A'C'}$$ then the triangles are similar. To show this, you should mimic the proof of theorem 1. More concretely, consider consider the triangle $\triangle ABC$ and mark the point $M$ on the ray $AB$ such that $AM = A'B'$. Then draw a line through $M$ parallel to $BC$, and let $N$ be the intersection of this line with the ray $AC$. Show that $\triangle ABC \sim \triangle AMN$, then use the intercept theorem to show that $AN = A'C'$. From there deduce that $\triangle AMN = \triangle A'B'C'$, and conclude the proof.
\end{prb}

\begin{prb}
In this problem you may use both theorem 1 and theorem 2, even if you did not solve problem 1.\\
\part{a} Suppose $\triangle ABC$ and $\triangle A'B'C'$ are similar. Let $AD$ and $A'D'$ be the angle bisectors of each triangle. Show that $$\frac{AD}{A'D'} = \frac{AB}{A'B'}$$
\textit{Hint: Show that $\triangle BAD \sim \triangle B'A'D'$.}
\part{b} In the same setup, let $AM$ and $A'M'$ be the medians of each triangle. Show that $$\frac{AM}{A'M'} = \frac{AB}{A'B'}$$
\end{prb}

\begin{prb}
In a round-robin tournament with four teams you get 2 points for winning, 1 point for a draw and 0 points for losing. If team $A$ had 5 points, team $B$ had 2 points and team $C$ had 1 point, which place did team $D$ get?
\end{prb}

\begin{prb}
Let $n$ be a positive integer divisible by 500. Show that the sum of even divisors of $n$ is greater than the sum of odd divisors of $n$. 
\end{prb}

\begin{prb}
Ten positive integers are written on the board, and it turnes out that all their last digits are distinct and all their second-to-last digits are also distinct. Show that their sum cannot be a perfect square. 
\end{prb}


\end{document}