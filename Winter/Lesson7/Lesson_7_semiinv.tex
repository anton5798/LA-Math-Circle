\documentclass[a4paper,12pt]{article} 
\usepackage{amsmath}
\usepackage[retainorgcmds]{IEEEtrantools}
\usepackage{graphicx}
\usepackage{amsfonts}
\usepackage{amsthm}
\usepackage{centernot}
\usepackage{setspace}
\usepackage{mathtools}
\usepackage{amssymb}
\usepackage{bm}
\usepackage[mathscr]{euscript}
\usepackage{pictexwd,dcpic}
\usepackage{tikz}
\usepackage{tikz-cd}
\usepackage[margin=.7in]{geometry}
\usepackage{breqn}
\newtheoremstyle{perfect}% name
  {}%         Space above, empty = `usual value'
  {}%         Space below
  {}% Body font
  {}%         Indent amount (empty = no indent, \parindent = para indent)
  {\bfseries}% Thm head font
  {.}%        Punctuation after thm head
  {\newline}% Space after thm head: \newline = linebreak
  {}%         Thm head spec
\theoremstyle{perfect}
\newtheorem{lem}{Lemma}
\newtheorem{thm}{Theorem}
\newtheorem{dfn}{Definition}
\newtheorem{exm}{Example}
\newtheorem{prop}{Proposition}
\newtheorem{crl}{Corollary}
\newtheorem{rem}{Reminder}
\newtheorem{prb}{Problem}
\newtheorem{exe}{Exercise}
\makeatletter
\newenvironment{cprb}[1]
  {\count@\c@prb
   \global\c@prb#1
    \global\advance\c@prb\m@ne
   \prb}
  {\endprb
   \global\c@prb\count@}
\makeatother
\makeatletter
\newenvironment{cexe}[1]
  {\count@\c@exe
   \global\c@exe#1
    \global\advance\c@exe\m@ne
   \exe}
  {\endexe
   \global\c@exe\count@}
\makeatother
\newcommand{\varline}{0}
\newcommand{\gen}[1]{\left< #1 \right>}
\newcommand{\ngen}[1]{\gen{\gen{#1}}}
\newcommand{\ov}[1]{\,\overline{#1}}
\DeclareMathOperator{\tor}{Tor}
\DeclareMathOperator{\aut}{Aut}
\DeclareMathOperator{\inn}{Inn}
\DeclareMathOperator{\im}{im}
\DeclareMathOperator{\imm}{Im}
\DeclareMathOperator{\ad}{ad}
\DeclareMathOperator{\Ad}{Ad}
\DeclareMathOperator{\Sp}{Sp}
\DeclareMathOperator{\SO}{SO}
\DeclareMathOperator{\SL}{SL}
\DeclareMathOperator{\SU}{SU}
\DeclareMathOperator{\GL}{GL}
\DeclareMathOperator{\PGL}{PGL}
\DeclareMathOperator{\re}{Re}
\DeclareMathOperator{\Hom}{Hom}
\DeclareMathOperator{\sym}{Sym}
\DeclareMathOperator{\ind}{Ind}
\DeclareMathOperator{\res}{Res}
\DeclareMathOperator{\sgn}{sgn}
\DeclareMathOperator{\End}{End}
\DeclareMathOperator{\colim}{colim}
\DeclareMathOperator{\coker}{coker}
\DeclareMathOperator{\Tr}{Tr}
\DeclareMathOperator{\intr}{int}
\DeclareMathOperator{\extr}{ext}
\DeclareMathOperator{\chr}{char}
\DeclareMathOperator{\supp}{supp}
\DeclareMathOperator{\hol}{Hol}
\DeclareMathOperator{\spec}{Spec}
\renewcommand{\Re}{\re}
\renewcommand{\Im}{\imm}
\newcommand{\eps}{\varepsilon}
\newcommand{\Mor}{\text{Mor}}
\newcommand{\cir}[1]{\mathrlap{\bigcirc}{\;#1}}
\newcommand{\Z}{\mathbb{Z}}
\newcommand{\Q}{\mathbb{Q}}
\newcommand{\R}{\mathbb{R}}
\newcommand{\C}{\mathbb{C}}
\newcommand{\F}{\mathbb{F}}
\newcommand{\N}{\mathbb{N}}
\newcommand{\lnorm}{\vartriangleleft}
\newcommand{\rnorm}{\vartriangleright}
\newcommand{\id}{\text{id}}
\newcommand{\dd}[1]{\mathrm{d}{#1}}
\newcommand{\p}[1]{\left( #1 \right)}
\newcommand{\parder}[2]{\frac{\partial #1}{\partial #2}}
\newcommand{\legendre}[2]{\left(\frac{#1}{#2}\right)}
\makeatletter
\renewcommand\part[1]{
\ifnum\pdfstrcmp{\varline}{1}=0
    \vspace{.10in}\textbf{\\#1)}
  \else
    \textbf{#1)}
  \fi\renewcommand{\varline}{1}}
\makeatother
\makeatletter
\newcommand{\tpmod}[1]{{\@displayfalse\pmod{#1}}}
\makeatother
\renewcommand{\restriction}{\mathord{\upharpoonright}}\author{Konstantin Miagkov} 
\title{Lesson 7: Invariants and the Intercept Theorem}
\begin{document} 
%\setstretch{1}
\maketitle

\section{From Last Time}

\begin{cprb}{1}
\part{a} Consider an $n \times m$ table filled with integers. With one operation, you are allowed to take any row or column and and negate every number in that row/column. Show that it is possible to make sure every row and column has nonnegative sum using such operations. 
\part{b} Same problem with real numbers in the table, not integers.                                 
\end{cprb}

\begin{cprb}{2}
Consider $n$ segments on the plane with $2n$ distinct endpoints. The following process is performed: if two segments $AB$ and $CD$ intersect, we replace them by segments $AD$ and $BC$. Show that eventually no two segments will intersect.                        
\end{cprb}

\section{New Problems}

\begin{prb}[Intercept Theorem]
\part{a} Consider two rays $r, \ell$ out of point $O$ and distinct points $A_1, \ldots, A_n$ on $r$ such that $$A_1A_2 = A_2A_3  = \ldots = A_{n-1}A_n$$ Show that if $B_1, \ldots, B_n$ are points on $\ell$ such that the lines $A_iB_i$ are parallel to each other for all $1 \leq i \leq n$, then $$B_1B_2 = B_2B_3  = \ldots = B_{n-1}B_n$$ \textit{Hint: Use L6.5.}
\part{b} With $O, r, \ell$ as in part a), let $A, B, C$ be points on $r$ such that $AB/BC$ is an integer. Show that if $A', B', C'$ are points on $\ell$ such that $AA' \parallel BB' \parallel CC'$, then $A'B'/B'C' = AB/BC$.
\part{c} Same as part b), except $AB/BC$ is rational.
\end{prb}

\begin{prb}
Let $AB$ be a given segment, and $n$ be a positive integer. Use straightedge and compass to split $AB$ into $n$ equal parts. You may assume the result of problem 1a).\\
\textit{Hint: Construct a random auxiliary line $\ell$ through $A$ and points $A_0, \ldots, A_{n}$ on $\ell$ such that $A_0 = A$ and $$A_0A_1 = A_1A_2 = \ldots = A_{n-1}A_n$$ Now use 1a).}
\end{prb}

\begin{prb}
Let $ABCD$ be an arbitrary quadrilateral. If $M$, $N$, $P$, $Q$ are the midpoints of $AB, BC, CD, DA$ respectively, show that $MNPQ$ is a parallelogram.
\end{prb}



\end{document}