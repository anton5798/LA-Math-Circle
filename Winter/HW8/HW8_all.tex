\documentclass[a4paper,12pt]{article} 
\usepackage{amsmath}
\usepackage[retainorgcmds]{IEEEtrantools}
\usepackage{graphicx}
\usepackage{amsfonts}
\usepackage{amsthm}
\usepackage{centernot}
\usepackage{setspace}
\usepackage{mathtools}
\usepackage{amssymb}
\usepackage{bm}
\usepackage[mathscr]{euscript}
\usepackage{pictexwd,dcpic}
\usepackage{tikz}
\usepackage{tikz-cd}
\usepackage[margin=1in]{geometry}
\usepackage{breqn}
\newtheoremstyle{perfect}% name
  {}%         Space above, empty = `usual value'
  {}%         Space below
  {}% Body font
  {}%         Indent amount (empty = no indent, \parindent = para indent)
  {\bfseries}% Thm head font
  {.}%        Punctuation after thm head
  {\newline}% Space after thm head: \newline = linebreak
  {}%         Thm head spec
\theoremstyle{perfect}
\newtheorem{lem}{Lemma}
\newtheorem{thm}{Theorem}
\newtheorem{dfn}{Definition}
\newtheorem{exm}{Example}
\newtheorem{prop}{Proposition}
\newtheorem{crl}{Corollary}
\newtheorem{rem}{Reminder}
\newtheorem{prb}{Problem}
\newtheorem{exe}{Exercise}
\makeatletter
\newenvironment{cprb}[1]
  {\count@\c@prb
   \global\c@prb#1
    \global\advance\c@prb\m@ne
   \prb}
  {\endprb
   \global\c@prb\count@}
\makeatother
\makeatletter
\newenvironment{cexe}[1]
  {\count@\c@exe
   \global\c@exe#1
    \global\advance\c@exe\m@ne
   \exe}
  {\endexe
   \global\c@exe\count@}
\makeatother
\newcommand{\varline}{0}
\newcommand{\gen}[1]{\left< #1 \right>}
\newcommand{\ngen}[1]{\gen{\gen{#1}}}
\newcommand{\ov}[1]{\,\overline{#1}}
\DeclareMathOperator{\tor}{Tor}
\DeclareMathOperator{\aut}{Aut}
\DeclareMathOperator{\inn}{Inn}
\DeclareMathOperator{\im}{im}
\DeclareMathOperator{\imm}{Im}
\DeclareMathOperator{\ad}{ad}
\DeclareMathOperator{\Ad}{Ad}
\DeclareMathOperator{\Sp}{Sp}
\DeclareMathOperator{\SO}{SO}
\DeclareMathOperator{\SL}{SL}
\DeclareMathOperator{\SU}{SU}
\DeclareMathOperator{\GL}{GL}
\DeclareMathOperator{\PGL}{PGL}
\DeclareMathOperator{\re}{Re}
\DeclareMathOperator{\Hom}{Hom}
\DeclareMathOperator{\sym}{Sym}
\DeclareMathOperator{\ind}{Ind}
\DeclareMathOperator{\res}{Res}
\DeclareMathOperator{\sgn}{sgn}
\DeclareMathOperator{\End}{End}
\DeclareMathOperator{\colim}{colim}
\DeclareMathOperator{\coker}{coker}
\DeclareMathOperator{\Tr}{Tr}
\DeclareMathOperator{\intr}{int}
\DeclareMathOperator{\extr}{ext}
\DeclareMathOperator{\chr}{char}
\DeclareMathOperator{\supp}{supp}
\DeclareMathOperator{\hol}{Hol}
\DeclareMathOperator{\spec}{Spec}
\renewcommand{\Re}{\re}
\renewcommand{\Im}{\imm}
\newcommand{\eps}{\varepsilon}
\newcommand{\Mor}{\text{Mor}}
\newcommand{\cir}[1]{\mathrlap{\bigcirc}{\;#1}}
\newcommand{\Z}{\mathbb{Z}}
\newcommand{\Q}{\mathbb{Q}}
\newcommand{\R}{\mathbb{R}}
\newcommand{\C}{\mathbb{C}}
\newcommand{\F}{\mathbb{F}}
\newcommand{\N}{\mathbb{N}}
\newcommand{\lnorm}{\vartriangleleft}
\newcommand{\rnorm}{\vartriangleright}
\newcommand{\id}{\text{id}}
\newcommand{\dd}[1]{\mathrm{d}{#1}}
\newcommand{\p}[1]{\left( #1 \right)}
\newcommand{\parder}[2]{\frac{\partial #1}{\partial #2}}
\newcommand{\legendre}[2]{\left(\frac{#1}{#2}\right)}
\makeatletter
\renewcommand\part[1]{
\ifnum\pdfstrcmp{\varline}{1}=0
    \vspace{.10in}\textbf{\\#1)}
  \else
    \textbf{#1)}
  \fi\renewcommand{\varline}{1}}
\makeatother
\makeatletter
\newcommand{\tpmod}[1]{{\@displayfalse\pmod{#1}}}
\makeatother
\renewcommand{\restriction}{\mathord{\upharpoonright}}\author{Konstantin Miagkov} 
\title{Homework 8: Similar Triangles}
\begin{document} 
%\setstretch{1}
\maketitle

\noindent In this homework, you have only one task. We would like you to write out a complete derivation of the main theorems about similar triangles, starting only with properties of congruent triangles, parallelograms and other objects we have studied previously. In essence, your text should be a combination of the solutions to L5.5, L6.5, L7.1 and proofs of theorems 1, 2 from lesson 8. Try as best as you can to make your writing a single coherent text which would serve as a good introduction to similar triangles to the reader previously unfamiliar with them. All the steps and problems you have to use have already been discussed in class at the board. Below we repeat the problem statements of the problems you need to use as steps for your convenience. You are allowed to simply state that the intercept theorem for rational ratios implies one for all real rations without further explanation. 

\begin{prb}[L5.5]
Let $M$ be the midpoint of the side $AB$ of $\triangle ABC$. Consider the line through $M$ which is parallel to $AC$, and suppose it intersects $BC$ at point $P$. Show that $P$ is the midpoint of $BC$. \textit{Hint: consider the line through $P$ parallel to $AB$ and its intersection with $AC$. }
\end{prb}

\begin{prb}[L6.5]
Consider two rays $r, \ell$ out of point $O$, a segment $AB$ on $r$ and point $C$ on $\ell$. Let $M$ be the midpoint of $AB$, let $D$ be the intersection of $\ell$ and the line through $M$ parallel to $AC$ and let $E$ be the intersection of $\ell$ and the line through $B$ parallel to $AC$. Show that $D$ is the midpoint of $CE$. 
\end{prb}

\begin{prb}[L7.1, Intercept Theorem]
\part{a} Consider two rays $r, \ell$ out of point $O$ and distinct points $A_1, \ldots, A_n$ on $r$ such that $$A_1A_2 = A_2A_3  = \ldots = A_{n-1}A_n$$ Show that if $B_1, \ldots, B_n$ are points on $\ell$ such that the lines $A_iB_i$ are parallel to each other for all $1 \leq i \leq n$, then $$B_1B_2 = B_2B_3  = \ldots = B_{n-1}B_n$$ \textit{Hint: We proved the case $n=2$ last week in problem L6.5. For the general case, use the statement for $n=2$ to show that $B_1B_2 = B_2B_3$, then use it to show that $B_2B_3 = B_3B_4$, and so on to conclude the problem.}
\part{b} With $O, r, \ell$ as in part a), let $A, B, C$ be points on $r$ such that $AB/BC$ is an integer. Show that if $A', B', C'$ are points on $\ell$ such that $AA' \parallel BB' \parallel CC'$, then $A'B'/B'C' = AB/BC$.\\
\textit{Hint: let $AB/BC = n$. Set up points $A_1, \ldots, A_{n-1}$ on $AB$ such that $$AA_1  = A_1A_2 = \ldots = A_{n-1}B = BC$$ then use part a).}
\part{c} Same as part b), except $AB/BC$ is rational.\\
\textit{Hint: if $AB/BC = m/n$, add some extra points on both the segment $AB$ and the segment $BC$ similarly to part b), then use 1a).}
\end{prb}

\begin{thm}[Lesson 8, Theorem 1]
Triangles $\triangle ABC$ and $\triangle A'B'C'$ are similar if and only if $$\frac{AB}{A'B'} = \frac{BC}{B'C'} = \frac{AC}{A'C'}$$
In other words, they have the same ratio of sides. This common ratio is called the \textit{similarity ratio}.
\end{thm}
\begin{thm}[Lesson 8, Theorem 2]
Triangles $\triangle ABC$ and $\triangle A'B'C'$ are similar if and only if $\angle BAC = \angle B'A'C'$ and $$\frac{AB}{A'B'} = \frac{AC}{A'C'}$$
In other words, they share an angle and have the same ratio of sides adjacent to that angle.
\end{thm}
\begin{prb}[L8.1]
In lecture we proved theorem 1. Now you need to prove theorem 2. Here are the suggested steps. If the triangles are already similar, you may use the already proven theorem 1 or the intercept theorem to get the condition on the ratios of sides. The more difficult direction is to show that if $\angle BAC = \angle B'A'C'$ and $$\frac{AB}{A'B'} = \frac{AC}{A'C'}$$ then the triangles are similar. To show this, you should mimic the proof of theorem 1. More concretely, consider consider the triangle $\triangle ABC$ and mark the point $M$ on the ray $AB$ such that $AM = A'B'$. Then draw a line through $M$ parallel to $BC$, and let $N$ be the intersection of this line with the ray $AC$. Show that $\triangle ABC \sim \triangle AMN$, then use the intercept theorem to show that $AN = A'C'$. From there deduce that $\triangle AMN = \triangle A'B'C'$, and conclude the proof.
\end{prb}


\end{document}