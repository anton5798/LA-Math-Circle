\documentclass[a4paper,12pt]{article} 
\usepackage{amsmath}
\usepackage[retainorgcmds]{IEEEtrantools}
\usepackage{graphicx}
\usepackage{amsfonts}
\usepackage{amsthm}
\usepackage{centernot}
\usepackage{setspace}
\usepackage{mathtools}
\usepackage{amssymb}
\usepackage{bm}
\usepackage[mathscr]{euscript}
\usepackage{pictexwd,dcpic}
\usepackage{tikz}
\usepackage{tikz-cd}
%\usepackage[margin=1in]{geometry}
\usepackage{breqn}
\newtheoremstyle{perfect}% name
  {}%         Space above, empty = `usual value'
  {}%         Space below
  {}% Body font
  {}%         Indent amount (empty = no indent, \parindent = para indent)
  {\bfseries}% Thm head font
  {.}%        Punctuation after thm head
  {\newline}% Space after thm head: \newline = linebreak
  {}%         Thm head spec
\theoremstyle{perfect}
\newtheorem{lem}{Lemma}
\newtheorem{thm}{Theorem}
\newtheorem{dfn}{Definition}
\newtheorem{exm}{Example}
\newtheorem{prop}{Proposition}
\newtheorem{crl}{Corollary}
\newtheorem{rem}{Reminder}
\newtheorem{prb}{Problem}
\newtheorem{exe}{Exercise}
\makeatletter
\newenvironment{cprb}[1]
  {\count@\c@prb
   \global\c@prb#1
    \global\advance\c@prb\m@ne
   \prb}
  {\endprb
   \global\c@prb\count@}
\makeatother
\makeatletter
\newenvironment{cexe}[1]
  {\count@\c@exe
   \global\c@exe#1
    \global\advance\c@exe\m@ne
   \exe}
  {\endexe
   \global\c@exe\count@}
\makeatother
\newcommand{\varline}{0}
\newcommand{\gen}[1]{\left< #1 \right>}
\newcommand{\ngen}[1]{\gen{\gen{#1}}}
\newcommand{\ov}[1]{\,\overline{#1}}
\DeclareMathOperator{\tor}{Tor}
\DeclareMathOperator{\aut}{Aut}
\DeclareMathOperator{\inn}{Inn}
\DeclareMathOperator{\im}{im}
\DeclareMathOperator{\imm}{Im}
\DeclareMathOperator{\ad}{ad}
\DeclareMathOperator{\Ad}{Ad}
\DeclareMathOperator{\Sp}{Sp}
\DeclareMathOperator{\SO}{SO}
\DeclareMathOperator{\SL}{SL}
\DeclareMathOperator{\SU}{SU}
\DeclareMathOperator{\GL}{GL}
\DeclareMathOperator{\PGL}{PGL}
\DeclareMathOperator{\re}{Re}
\DeclareMathOperator{\Hom}{Hom}
\DeclareMathOperator{\sym}{Sym}
\DeclareMathOperator{\ind}{Ind}
\DeclareMathOperator{\res}{Res}
\DeclareMathOperator{\sgn}{sgn}
\DeclareMathOperator{\End}{End}
\DeclareMathOperator{\colim}{colim}
\DeclareMathOperator{\coker}{coker}
\DeclareMathOperator{\Tr}{Tr}
\DeclareMathOperator{\intr}{int}
\DeclareMathOperator{\extr}{ext}
\DeclareMathOperator{\chr}{char}
\DeclareMathOperator{\supp}{supp}
\DeclareMathOperator{\hol}{Hol}
\DeclareMathOperator{\spec}{Spec}
\renewcommand{\Re}{\re}
\renewcommand{\Im}{\imm}
\newcommand{\eps}{\varepsilon}
\newcommand{\Mor}{\text{Mor}}
\newcommand{\cir}[1]{\mathrlap{\bigcirc}{\;#1}}
\newcommand{\Z}{\mathbb{Z}}
\newcommand{\Q}{\mathbb{Q}}
\newcommand{\R}{\mathbb{R}}
\newcommand{\C}{\mathbb{C}}
\newcommand{\F}{\mathbb{F}}
\newcommand{\N}{\mathbb{N}}
\newcommand{\lnorm}{\vartriangleleft}
\newcommand{\rnorm}{\vartriangleright}
\newcommand{\id}{\text{id}}
\newcommand{\dd}[1]{\mathrm{d}{#1}}
\newcommand{\p}[1]{\left( #1 \right)}
\newcommand{\parder}[2]{\frac{\partial #1}{\partial #2}}
\newcommand{\legendre}[2]{\left(\frac{#1}{#2}\right)}
\makeatletter
\renewcommand\part[1]{
\ifnum\pdfstrcmp{\varline}{1}=0
    \vspace{.10in}\textbf{\\#1)}
  \else
    \textbf{#1)}
  \fi\renewcommand{\varline}{1}}
\makeatother
\makeatletter
\newcommand{\tpmod}[1]{{\@displayfalse\pmod{#1}}}
\makeatother
\renewcommand{\restriction}{\mathord{\upharpoonright}}\author{Konstantin Miagkov} 
\title{Combinatorics and Ruler \& Compass Constructions}
\begin{document} 
%\setstretch{1}
\maketitle

\begin{prb}
You are given a balance scale and a weight of 1 gram. Can you use them to measure out 1 kilogram of sugar by using the scale at most 10 times?
\end{prb}

\begin{prb}
There are 68 coins on the table, and any two coins weigh differently. Show how to determine the heaviest and the lightest coins using a balance scale at most 100 times. 
\end{prb}


We will explore the island of knights, knaves and spies, where there are three types of people: knights, who always tell the truth; knaves, who always lie; and spies, who can lie or tell the truth at will.

\begin{prb}
On the island of knights, knaves and spies, you come across three people. One wears blue, one wears red, and one wears green. You know that one is a knight, one is a knave, and one is a spy. 
``Who is the spy?" you ask.

\begin{itemize}
\item The man wearing blue says, ``That man in red is the spy."
\item The man wearing red says, ``No, the man in green is the spy."
\item The man wearing green says, ``No, the man in red is in fact the spy."
\end{itemize}
\noindent
Who is the spy? Who is the knight and who is the knave?
\end{prb}

\begin{prb}
There are 30 knights and knaves sitting at a round table. Each person has exactly one friend. Friend of a knight is always a knave, and friend of a knave is always a knight (friendship is mutual). You ask: ``Is your friend sitting next to you?" and receive answers ``Yes" from 15 people. How many other people could have possibly answered ``Yes" as well?
\end{prb}


\begin{prb}
\part{a} Given a segment $AB$ on the plane, construct a point $C$ such that $ABC$ is an equilateral triangle using the ruler and compass.
\part{b} Construct the midpoint of the segment $AB$. 
\end{prb}

\begin{prb}
Given a point $A$ and two rays out of it forming an angle $\alpha$, construct the angle bisector of $\alpha$. 
\end{prb}


\end{document}