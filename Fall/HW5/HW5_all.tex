\documentclass[a4paper,12pt]{article} 
\usepackage{amsmath}
\usepackage[retainorgcmds]{IEEEtrantools}
\usepackage{graphicx}
\usepackage{amsfonts}
\usepackage{amsthm}
\usepackage{centernot}
\usepackage{setspace}
\usepackage{mathtools}
\usepackage{amssymb}
\usepackage{bm}
\usepackage[mathscr]{euscript}
\usepackage{pictexwd,dcpic}
\usepackage{tikz}
\usepackage{tikz-cd}
\usepackage[margin=1in]{geometry}
\usepackage{breqn}
\newtheoremstyle{perfect}% name
  {}%         Space above, empty = `usual value'
  {}%         Space below
  {}% Body font
  {}%         Indent amount (empty = no indent, \parindent = para indent)
  {\bfseries}% Thm head font
  {.}%        Punctuation after thm head
  {\newline}% Space after thm head: \newline = linebreak
  {}%         Thm head spec
\theoremstyle{perfect}
\newtheorem{lem}{Lemma}
\newtheorem{thm}{Theorem}
\newtheorem{dfn}{Definition}
\newtheorem{exm}{Example}
\newtheorem{prop}{Proposition}
\newtheorem{crl}{Corollary}
\newtheorem{sol}{Solution}
\newtheorem{rem}{Reminder}
\newtheorem{prb}{Problem}
\newtheorem{exe}{Exercise}
\makeatletter
\newenvironment{cprb}[1]
  {\count@\c@prb
   \global\c@prb#1
    \global\advance\c@prb\m@ne
   \prb}
  {\endprb
   \global\c@prb\count@}
\makeatother
\makeatletter
\newenvironment{cexe}[1]
  {\count@\c@exe
   \global\c@exe#1
    \global\advance\c@exe\m@ne
   \exe}
  {\endexe
   \global\c@exe\count@}
\makeatother
\newcommand{\varline}{0}
\newcommand{\gen}[1]{\left< #1 \right>}
\newcommand{\ngen}[1]{\gen{\gen{#1}}}
\newcommand{\ov}[1]{\,\overline{#1}}
\DeclareMathOperator{\tor}{Tor}
\DeclareMathOperator{\aut}{Aut}
\DeclareMathOperator{\inn}{Inn}
\DeclareMathOperator{\im}{im}
\DeclareMathOperator{\imm}{Im}
\DeclareMathOperator{\ad}{ad}
\DeclareMathOperator{\Ad}{Ad}
\DeclareMathOperator{\Sp}{Sp}
\DeclareMathOperator{\SO}{SO}
\DeclareMathOperator{\SL}{SL}
\DeclareMathOperator{\SU}{SU}
\DeclareMathOperator{\GL}{GL}
\DeclareMathOperator{\PGL}{PGL}
\DeclareMathOperator{\re}{Re}
\DeclareMathOperator{\Hom}{Hom}
\DeclareMathOperator{\sym}{Sym}
\DeclareMathOperator{\ind}{Ind}
\DeclareMathOperator{\res}{Res}
\DeclareMathOperator{\sgn}{sgn}
\DeclareMathOperator{\End}{End}
\DeclareMathOperator{\colim}{colim}
\DeclareMathOperator{\coker}{coker}
\DeclareMathOperator{\Tr}{Tr}
\DeclareMathOperator{\intr}{int}
\DeclareMathOperator{\extr}{ext}
\DeclareMathOperator{\chr}{char}
\DeclareMathOperator{\supp}{supp}
\DeclareMathOperator{\hol}{Hol}
\DeclareMathOperator{\spec}{Spec}
\renewcommand{\Re}{\re}
\renewcommand{\Im}{\imm}
\newcommand{\eps}{\varepsilon}
\newcommand{\Mor}{\text{Mor}}
\newcommand{\cir}[1]{\mathrlap{\bigcirc}{\;#1}}
\newcommand{\Z}{\mathbb{Z}}
\newcommand{\Q}{\mathbb{Q}}
\newcommand{\R}{\mathbb{R}}
\newcommand{\C}{\mathbb{C}}
\newcommand{\F}{\mathbb{F}}
\newcommand{\N}{\mathbb{N}}
\newcommand{\lnorm}{\vartriangleleft}
\newcommand{\rnorm}{\vartriangleright}
\newcommand{\id}{\text{id}}
\newcommand{\dd}[1]{\mathrm{d}{#1}}
\newcommand{\p}[1]{\left( #1 \right)}
\newcommand{\parder}[2]{\frac{\partial #1}{\partial #2}}
\newcommand{\legendre}[2]{\left(\frac{#1}{#2}\right)}
\makeatletter
\renewcommand\part[1]{
\ifnum\pdfstrcmp{\varline}{1}=0
    \vspace{.10in}\textbf{\\#1)}
  \else
    \textbf{#1)}
  \fi\renewcommand{\varline}{1}}
\makeatother
\makeatletter
\newcommand{\tpmod}[1]{{\@displayfalse\pmod{#1}}}
\makeatother
\renewcommand{\restriction}{\mathord{\upharpoonright}}\author{Konstantin Miagkov} 
\title{Homework 5: Quadratic Inequalities}
\begin{document} 
%\setstretch{1}
\maketitle

\section{Reading}

\begin{sol}[L3.4]
Suppose our quadratic equation is $ax^2+bx+c = 0$ and has roots $x_0, x_1$. Then by Vieta's Theorem we know that $b = -a(x_0+x_1)$ and $c = a(x_0x_1)$. Since all the number involved are integers, we know that $a \mid c$, $x_0 \mid c$ and $x_1 \mid c$. So $c$ is divisible by at least three of the other numbers. But among the four integers left on the board there is only one pair where one is divisible by another: $2 \mid 4$. This means that the erased number must have been $c$. Since Vieta's theorem also tells us that $a \mid b$, we must have $a = 2$ and $b = 4$. Then the roots are $3, -5$, and we can find $c$ via $c = a(x_0x_1) = 2 \cdot 3 \cdot (-5) = -30$, which is the answer
\end{sol}

\begin{sol}[L3.2b]
Suppose we want to write our function $f$ as $g+h$, where $g$ is even and $h$ is odd. This means that for any $x 
\in \R$ we have $$g(-x) = g(x)$$ $$h(-x) = -h(x)$$ $$f(x) = g(x)+h(x)$$ We also know that $$f(-x) = g(-x) + h(-x) = g(x) - h(x)$$ Adding the last two equations and subtracting we get $$f(x) + f(-x) = 2g(x)$$ $$f(x) - f(-x) = 2h(x)$$ or, after dividing by 2, $$\frac{f(x) + f(-x)}{2} = g(x)$$ $$\frac{f(x) - f(-x)}{2} = h(x)$$ This takes care of the uniqueness part -- we just showed that if such $g$ and $h$ do exist, they must be given exactly by formulas $$g(x) = \frac{f(x) + f(-x)}{2}$$ $$h(x) = \frac{f(x) - f(-x)}{2}$$ for all real $x$. On the other hand, this also gives us the existence. If we define $g$ and $h$ as above, then they are indeed even and odd: $$g(-x) = \frac{f(-x) + f(x)}{2} = \frac{f(x) + f(-x)}{2} = g(x)$$ $$h(-x) = \frac{f(-x) - f(x)}{2} = -\frac{f(x) - f(-x)}{2} = -h(x)$$ And for all $x \in \R$ we have $$g(x) + h(x) = \frac{f(x) + f(-x)}{2} + \frac{f(x) - f(-x)}{2} = f(x)$$ so we are done.
 \end{sol}

\section{Homework}

\begin{prb}
Without computing the roots $x_0, x_1$ of the equation $3x^2+8x-1$, determine the following quantities:\\
\part{a} $x_0x_1^4 + x_1x_0^4$
\part{b} $x_0^4 + x_1^4$
\part{c} $x_0/x_1+x_1/x_0$
\end{prb}

\begin{prb}
The quadrilateral $ABCD$ is inscribed and circumscribed at the same time, and the centers of its incircle and circumcircle coincide. Show that $ABCD$ is a square.
\end{prb}

\end{document}