\documentclass[a4paper,12pt]{article} 
\usepackage{amsmath}
\usepackage[retainorgcmds]{IEEEtrantools}
\usepackage{graphicx}
\usepackage{amsfonts}
\usepackage{amsthm}
\usepackage{centernot}
\usepackage{setspace}
\usepackage{mathtools}
\usepackage{amssymb}
\usepackage{bm}
\usepackage[mathscr]{euscript}
\usepackage{pictexwd,dcpic}
\usepackage{tikz}
\usepackage{tikz-cd}
\usepackage[margin=1in]{geometry}
\usepackage{breqn}
\newtheoremstyle{perfect}% name
  {}%         Space above, empty = `usual value'
  {}%         Space below
  {}% Body font
  {}%         Indent amount (empty = no indent, \parindent = para indent)
  {\bfseries}% Thm head font
  {.}%        Punctuation after thm head
  {\newline}% Space after thm head: \newline = linebreak
  {}%         Thm head spec
\theoremstyle{perfect}
\newtheorem{lem}{Lemma}
\newtheorem{thm}{Theorem}
\newtheorem{dfn}{Definition}
\newtheorem{exm}{Example}
\newtheorem{prop}{Proposition}
\newtheorem{crl}{Corollary}
\newtheorem{rem}{Reminder}
\newtheorem{prb}{Problem}
\newtheorem{exe}{Exercise}
\makeatletter
\newenvironment{cprb}[1]
  {\count@\c@prb
   \global\c@prb#1
    \global\advance\c@prb\m@ne
   \prb}
  {\endprb
   \global\c@prb\count@}
\makeatother
\makeatletter
\newenvironment{cexe}[1]
  {\count@\c@exe
   \global\c@exe#1
    \global\advance\c@exe\m@ne
   \exe}
  {\endexe
   \global\c@exe\count@}
\makeatother
\newcommand{\varline}{0}
\newcommand{\gen}[1]{\left< #1 \right>}
\newcommand{\ngen}[1]{\gen{\gen{#1}}}
\newcommand{\ov}[1]{\,\overline{#1}}
\DeclareMathOperator{\tor}{Tor}
\DeclareMathOperator{\aut}{Aut}
\DeclareMathOperator{\inn}{Inn}
\DeclareMathOperator{\im}{im}
\DeclareMathOperator{\imm}{Im}
\DeclareMathOperator{\ad}{ad}
\DeclareMathOperator{\Ad}{Ad}
\DeclareMathOperator{\Sp}{Sp}
\DeclareMathOperator{\SO}{SO}
\DeclareMathOperator{\SL}{SL}
\DeclareMathOperator{\SU}{SU}
\DeclareMathOperator{\GL}{GL}
\DeclareMathOperator{\PGL}{PGL}
\DeclareMathOperator{\re}{Re}
\DeclareMathOperator{\Hom}{Hom}
\DeclareMathOperator{\sym}{Sym}
\DeclareMathOperator{\ind}{Ind}
\DeclareMathOperator{\res}{Res}
\DeclareMathOperator{\sgn}{sgn}
\DeclareMathOperator{\End}{End}
\DeclareMathOperator{\colim}{colim}
\DeclareMathOperator{\coker}{coker}
\DeclareMathOperator{\Tr}{Tr}
\DeclareMathOperator{\intr}{int}
\DeclareMathOperator{\extr}{ext}
\DeclareMathOperator{\chr}{char}
\DeclareMathOperator{\supp}{supp}
\DeclareMathOperator{\hol}{Hol}
\DeclareMathOperator{\spec}{Spec}
\renewcommand{\Re}{\re}
\renewcommand{\Im}{\imm}
\newcommand{\eps}{\varepsilon}
\newcommand{\Mor}{\text{Mor}}
\newcommand{\cir}[1]{\mathrlap{\bigcirc}{\;#1}}
\newcommand{\Z}{\mathbb{Z}}
\newcommand{\Q}{\mathbb{Q}}
\newcommand{\R}{\mathbb{R}}
\newcommand{\C}{\mathbb{C}}
\newcommand{\F}{\mathbb{F}}
\newcommand{\N}{\mathbb{N}}
\newcommand{\lnorm}{\vartriangleleft}
\newcommand{\rnorm}{\vartriangleright}
\newcommand{\id}{\text{id}}
\newcommand{\dd}[1]{\mathrm{d}{#1}}
\newcommand{\p}[1]{\left( #1 \right)}
\newcommand{\parder}[2]{\frac{\partial #1}{\partial #2}}
\newcommand{\legendre}[2]{\left(\frac{#1}{#2}\right)}
\makeatletter
\renewcommand\part[1]{
\ifnum\pdfstrcmp{\varline}{1}=0
    \vspace{.10in}\textbf{\\#1)}
  \else
    \textbf{#1)}
  \fi\renewcommand{\varline}{1}}
\makeatother
\makeatletter
\newcommand{\tpmod}[1]{{\@displayfalse\pmod{#1}}}
\makeatother
\renewcommand{\restriction}{\mathord{\upharpoonright}}\author{Konstantin Miagkov} 
\title{Lesson 3: Functions and Quadratic Equations}
\begin{document} 
%\setstretch{1}
\maketitle

\begin{dfn}
A \textbf{function} from set $X$ to set $Y$ is a rule that for each element $x \in X$ determines an element from the set $Y$, usually denoted $f(x)$.
\end{dfn}

\begin{dfn}
Let $f,g : \R \to \R$ be functions. We write $f > g \; (f \geq g)$ if for every $x \in \R: \; f(x) > g(x) \; (f(x) \geq g(x))$.
\end{dfn}

\begin{prb}
Determine if one (or none) of $f \geq g$, $f \leq  g$ holds:\\
\textbf{a)} $f(x) = 2x + 3$, $g(x) = 2x + 1$ \\
\textbf{b)} $f(x) = 2x^2$, $g(x) = 3x^2$  \\
\textbf{c)} $f(x) = x + 4$, $g(x) = x^2$\\
\textbf{d)} $f(x) = 2x + 1$, $g(x) = -x^2$
\end{prb}


\begin{prb}
A function $f : \R \to \R$ is called \textit{even} if $f(x) = f(-x)$ for all $x \in \R$. Similarly, a function is called \textit{odd} if $f(x)= -f(-x)$ for all $x$.\\
\part{a} Find which of the following functions are odd, even or neither: $x\cdot\lvert x \rvert$; $\lvert x+1\rvert - \lvert x-1\rvert$; $\lvert x+1\rvert + \lvert x-1\rvert$; $3x - x^2$.
\part{b} Show that any function from $\R$ to $\R$ can be uniquely written as a sum of an even and an odd function.
\end{prb}

\begin{prb}
Find all functions $f : \R \to \R$ such that $f(2x+1) = 4x^2+14x+7$.
\end{prb}

\begin{prb}
Five integers are written on the board -- three coefficients of a quadratic equation and two roots in arbitrary order. After one of the numbers is erased, the numbers $2,3,4,-5$ are left. What number was erased?
\end{prb}

\begin{prb}
Let $ABCD$ be a quadrilateral such that there exists a circle tangent to all of its four sides. Such a quadrilateral is called \textit{circumscribed}. Show that $AB + CD = AD + BC$.
\end{prb}



\end{document}