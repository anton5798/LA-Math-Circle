\documentclass[a4paper,12pt]{article} 
\usepackage{amsmath}
\usepackage[retainorgcmds]{IEEEtrantools}
\usepackage{graphicx}
\usepackage{amsfonts}
\usepackage{amsthm}
\usepackage{centernot}
\usepackage{setspace}
\usepackage{mathtools}
\usepackage{amssymb}
\usepackage{bm}
\usepackage[mathscr]{euscript}
\usepackage{pictexwd,dcpic}
\usepackage{tikz}
\usepackage{tikz-cd}
\usepackage[margin=1in]{geometry}
\usepackage{breqn}
\newtheoremstyle{perfect}% name
  {}%         Space above, empty = `usual value'
  {}%         Space below
  {}% Body font
  {}%         Indent amount (empty = no indent, \parindent = para indent)
  {\bfseries}% Thm head font
  {.}%        Punctuation after thm head
  {\newline}% Space after thm head: \newline = linebreak
  {}%         Thm head spec
\theoremstyle{perfect}
\newtheorem{lem}{Lemma}
\newtheorem{thm}{Theorem}
\newtheorem{dfn}{Definition}
\newtheorem{exm}{Example}
\newtheorem{prop}{Proposition}
\newtheorem{crl}{Corollary}
\newtheorem{sol}{Solution}
\newtheorem{rem}{Reminder}
\newtheorem{prb}{Problem}
\newtheorem{exe}{Exercise}
\makeatletter
\newenvironment{cprb}[1]
  {\count@\c@prb
   \global\c@prb#1
    \global\advance\c@prb\m@ne
   \prb}
  {\endprb
   \global\c@prb\count@}
\makeatother
\makeatletter
\newenvironment{cexe}[1]
  {\count@\c@exe
   \global\c@exe#1
    \global\advance\c@exe\m@ne
   \exe}
  {\endexe
   \global\c@exe\count@}
\makeatother
\newcommand{\varline}{0}
\newcommand{\gen}[1]{\left< #1 \right>}
\newcommand{\ngen}[1]{\gen{\gen{#1}}}
\newcommand{\ov}[1]{\,\overline{#1}}
\DeclareMathOperator{\tor}{Tor}
\DeclareMathOperator{\aut}{Aut}
\DeclareMathOperator{\inn}{Inn}
\DeclareMathOperator{\im}{im}
\DeclareMathOperator{\imm}{Im}
\DeclareMathOperator{\ad}{ad}
\DeclareMathOperator{\Ad}{Ad}
\DeclareMathOperator{\Sp}{Sp}
\DeclareMathOperator{\SO}{SO}
\DeclareMathOperator{\SL}{SL}
\DeclareMathOperator{\SU}{SU}
\DeclareMathOperator{\GL}{GL}
\DeclareMathOperator{\PGL}{PGL}
\DeclareMathOperator{\re}{Re}
\DeclareMathOperator{\Hom}{Hom}
\DeclareMathOperator{\sym}{Sym}
\DeclareMathOperator{\ind}{Ind}
\DeclareMathOperator{\res}{Res}
\DeclareMathOperator{\sgn}{sgn}
\DeclareMathOperator{\End}{End}
\DeclareMathOperator{\colim}{colim}
\DeclareMathOperator{\coker}{coker}
\DeclareMathOperator{\Tr}{Tr}
\DeclareMathOperator{\intr}{int}
\DeclareMathOperator{\extr}{ext}
\DeclareMathOperator{\chr}{char}
\DeclareMathOperator{\supp}{supp}
\DeclareMathOperator{\hol}{Hol}
\DeclareMathOperator{\spec}{Spec}
\renewcommand{\Re}{\re}
\renewcommand{\Im}{\imm}
\newcommand{\eps}{\varepsilon}
\newcommand{\Mor}{\text{Mor}}
\newcommand{\cir}[1]{\mathrlap{\bigcirc}{\;#1}}
\newcommand{\Z}{\mathbb{Z}}
\newcommand{\Q}{\mathbb{Q}}
\newcommand{\R}{\mathbb{R}}
\newcommand{\C}{\mathbb{C}}
\newcommand{\F}{\mathbb{F}}
\newcommand{\N}{\mathbb{N}}
\newcommand{\lnorm}{\vartriangleleft}
\newcommand{\rnorm}{\vartriangleright}
\newcommand{\id}{\text{id}}
\newcommand{\dd}[1]{\mathrm{d}{#1}}
\newcommand{\p}[1]{\left( #1 \right)}
\newcommand{\parder}[2]{\frac{\partial #1}{\partial #2}}
\newcommand{\legendre}[2]{\left(\frac{#1}{#2}\right)}
\makeatletter
\renewcommand\part[1]{
\ifnum\pdfstrcmp{\varline}{1}=0
    \vspace{.10in}\textbf{\\#1)}
  \else
    \textbf{#1)}
  \fi\renewcommand{\varline}{1}}
\makeatother
\makeatletter
\newcommand{\tpmod}[1]{{\@displayfalse\pmod{#1}}}
\makeatother
\renewcommand{\restriction}{\mathord{\upharpoonright}}\author{Konstantin Miagkov} 
\title{Homework 7: Quadratic equations VI}
\begin{document} 
%\setstretch{1}
\maketitle

\section{Reading}

\begin{sol}[H5.1]
In all parts, we will use Vieta's theorem, which tells us that $x_0+x_1 = -8/3$ and $x_0x_1 = -1/3$. Then we will try to rewrite the expressions in terms of $x_0+x_1$ and $x_0x_1$.\\
\part{a} \begin{align*}x_0x_1^4 + x_1x_0^4 & = x_0x_1(x_0^3+x_1^3) = x_0x_1((x_0+x_1)^3 - 3x_0x_1^2 - 3x_0^2x_1) \\ & = x_0x_1((x_0+x_1)^3 - 3x_0x_1(x_0+x_1)) = -\frac{1}{3}\left( -\frac{8^3}{3^3} - 3\cdot \frac{1}{3} \cdot \frac{8}{3} \right) = \frac{584}{81} \end{align*}
\part{b} \begin{align*}x_0^4 + x_1^4 & = (x_0+x_1)^4 - 4x_0x_1^3 - 6x_0^2x_1^2 - 4x_0^3x_1 \\ & = (x_0+x_1)^4 - 4x_0x_1((x_0+x_1)^2 - 2x_0x_1) - 6(x_0x_1)^2 \\ & =  \frac{8^4}{3^4} + 4\cdot \frac{1}{3}\left( \frac{8^2}{3^2} + 2\cdot \frac{1}{3} \right) - 6\frac{1^2}{3^2} = \frac{4882}{81} \end{align*}
\part{c} Notice that $0$ is not a root of our equation, so dividing by $x_0$ and $x_1$ is valid. \begin{align*}\frac{x_0}{x_1} + \frac{x_1}{x_0} = \frac{x_0^2 + x_1^2}{x_0x_1} = \frac{(x_0+x_1)^2 - 2x_0x_1}{x_0x_1} = -3\left(\frac{8^2}{3^2} + 2\cdot \frac{1}{3}\right) = -\frac{70}{3}\end{align*}
\end{sol}

\begin{sol}[H5.2]
Let $O$ be the common center of the incircle and the circumcircle of $ABCD$. Let $P, Q, R, S$ be the points at which the incircle is tangent to the sides $AB, BC, CD, DA$ respectively. Then $OP$ is the altitude in the isosceles $\triangle AOB$, which means that $AP = BP$. But we also know that $BP = BQ$. Applying the same logic to each side we get $AP = PB = BQ = QC = CR = RD = DS = SA$. But this implies that $AB = BC = CD = DA$, so $ABCD$ is a rhombus. But we know that in a rhombus $\angle A = \angle C$ and $\angle B = \angle D$, and the condition of being inscribed means that $\angle A + \angle C = \angle B + \angle D = 180^\circ$. This implies that $\angle A = \angle B = \angle C = \angle D = 90^\circ$, which means that our rhombus is in fact a square.
 \end{sol}

\section{Homework}

\begin{prb}
The quadratic equation $x^2+bx+c$ has two distinct real roots. Is it possible that after each coefficient was increased by 1, each root also increased by 1?
\end{prb}

\begin{prb}
Let $ABCD$ be a convex quadrilateral. Show that the quadrilateral created by the four angle bisectors of $ABCD$ is inscribed.
\end{prb}




\end{document}