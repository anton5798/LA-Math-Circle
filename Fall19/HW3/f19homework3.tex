\documentclass[a4paper,12pt]{article} 
\usepackage{amsmath}
\usepackage[retainorgcmds]{IEEEtrantools}
\usepackage{graphicx}
\usepackage{amsfonts}
\usepackage{amsthm}
\usepackage{centernot}
\usepackage{setspace}
\usepackage{mathtools}
\usepackage{amssymb}
\usepackage{bm}
\usepackage[mathscr]{euscript}
\usepackage{pictexwd,dcpic}
\usepackage{tikz}
\usepackage{tikz-cd}
\usepackage[margin=1in]{geometry}
\usepackage{breqn}
\newtheoremstyle{perfect}% name
  {}%         Space above, empty = `usual value'
  {}%         Space below
  {}% Body font
  {}%         Indent amount (empty = no indent, \parindent = para indent)
  {\bfseries}% Thm head font
  {.}%        Punctuation after thm head
  {\newline}% Space after thm head: \newline = linebreak
  {}%         Thm head spec
\theoremstyle{perfect}
\newtheorem{lem}{Lemma}
\newtheorem{thm}{Theorem}
\newtheorem{dfn}{Definition}
\newtheorem{exm}{Example}
\newtheorem{prop}{Proposition}
\newtheorem{crl}{Corollary}
\newtheorem{rem}{Reminder}
\newtheorem{prb}{Problem}
\newtheorem{sol}{Solution}
\newtheorem{exe}{Exercise}
\makeatletter
\newenvironment{csol}[1]
  {\count@\c@sol
   \global\c@sol#1
    \global\advance\c@sol\m@ne
   \sol}
  {\endsol
   \global\c@sol\count@}
\makeatother

\makeatletter
\newenvironment{cprb}[1]
  {\count@\c@prb
   \global\c@prb#1
    \global\advance\c@prb\m@ne
   \prb}
  {\endprb
   \global\c@prb\count@}
\makeatother
\makeatletter
\newenvironment{cexe}[1]
  {\count@\c@exe
   \global\c@exe#1
    \global\advance\c@exe\m@ne
   \exe}
  {\endexe
   \global\c@exe\count@}
\makeatother
\newcommand{\varline}{0}
\newcommand{\gen}[1]{\left< #1 \right>}
\newcommand{\ngen}[1]{\gen{\gen{#1}}}
\newcommand{\ov}[1]{\,\overline{#1}}
\DeclareMathOperator{\tor}{Tor}
\DeclareMathOperator{\aut}{Aut}
\DeclareMathOperator{\inn}{Inn}
\DeclareMathOperator{\im}{im}
\DeclareMathOperator{\imm}{Im}
\DeclareMathOperator{\ad}{ad}
\DeclareMathOperator{\Ad}{Ad}
\DeclareMathOperator{\Sp}{Sp}
\DeclareMathOperator{\SO}{SO}
\DeclareMathOperator{\SL}{SL}
\DeclareMathOperator{\SU}{SU}
\DeclareMathOperator{\GL}{GL}
\DeclareMathOperator{\PGL}{PGL}
\DeclareMathOperator{\re}{Re}
\DeclareMathOperator{\Hom}{Hom}
\DeclareMathOperator{\sym}{Sym}
\DeclareMathOperator{\ind}{Ind}
\DeclareMathOperator{\res}{Res}
\DeclareMathOperator{\sgn}{sgn}
\DeclareMathOperator{\End}{End}
\DeclareMathOperator{\colim}{colim}
\DeclareMathOperator{\coker}{coker}
\DeclareMathOperator{\Tr}{Tr}
\DeclareMathOperator{\intr}{int}
\DeclareMathOperator{\extr}{ext}
\DeclareMathOperator{\chr}{char}
\DeclareMathOperator{\supp}{supp}
\DeclareMathOperator{\hol}{Hol}
\DeclareMathOperator{\spec}{Spec}
\renewcommand{\Re}{\re}
\renewcommand{\Im}{\imm}
\newcommand{\eps}{\varepsilon}
\newcommand{\Mor}{\text{Mor}}
\newcommand{\cir}[1]{\mathrlap{\bigcirc}{\;#1}}
\newcommand{\Z}{\mathbb{Z}}
\newcommand{\Q}{\mathbb{Q}}
\newcommand{\R}{\mathbb{R}}
\newcommand{\C}{\mathbb{C}}
\newcommand{\F}{\mathbb{F}}
\newcommand{\N}{\mathbb{N}}
\newcommand{\PP}{\mathbf{P}}
\newcommand{\lnorm}{\vartriangleleft}
\newcommand{\rnorm}{\vartriangleright}
\newcommand{\id}{\text{id}}
\newcommand{\dd}[1]{\mathrm{d}{#1}}
\newcommand{\p}[1]{\left( #1 \right)}
\newcommand{\parder}[2]{\frac{\partial #1}{\partial #2}}
\newcommand{\legendre}[2]{\left(\frac{#1}{#2}\right)}
\makeatletter
\renewcommand\part[1]{
\ifnum\pdfstrcmp{\varline}{1}=0
    \vspace{.10in}\textbf{\\#1)}
  \else
    \textbf{#1)}
  \fi\renewcommand{\varline}{1}}
\makeatother
\makeatletter
\newcommand{\tpmod}[1]{{\@displayfalse\pmod{#1}}}
\makeatother
\renewcommand{\restriction}{\mathord{\upharpoonright}}\author{Anton Lykov} 
\title{Solutions \& Homework 3}
\begin{document} 
%\setstretch{1}
\maketitle

\section{Solutions}

\begin{cprb}{2}
Calculate the probability of flipping a fair coin three times and getting 2H and 1T \\
\textbf{a)} In this order \\
\textbf{b)} In any order 
\end{cprb}

\begin{csol}{2}
The sample space in this problem looks like this: 
$$\Omega = \{ \text{HHH,  HHT, HTH, THH, HTT, THT, TTH, TTT}\}$$
where TTH, for instance, means that the first $2$ flips were tails, and last one was heads. 
The probability of each of these elementary events is $\frac{1}{8}$. \\
\textbf{a)} The corresponding event is $\{ \text{HHT} \}$, and so the probability is $\frac{1}{8}$\\
\textbf{b)} The corresponding event is $\{ \text{HHT, HTH, THH} \}$, and so the probability is $\frac{3}{8}$.
\end{csol}

\begin{cprb}{4}
Calculate the probability of rolling a six-sided die three times and getting a sum of three outcomes equal to 10.
\end{cprb}

\begin{csol}{4}
There are $6^3$ possible elementary outcomes total, each one equally likely. We need to calculate how many of these outcomes belong to event $E = \{\text{sum of three rolls is 10}\}$, 
i.e. how many ways there are to represent $10$ as a sum of three integers from $1$ to $6$. 
There are $27$ ways of doing so, and thus the answer is $\frac{27}{6^3}$.
\end{csol}

\section{Homework}

\begin{prb}
\textbf{a)} Given a random $2$-digit integer, what is the probability that the sum of its digits is equal to $9$?
\\
\textbf{b)} The same question, but you additionally know that each digit is at least $3$.
\end{prb}

\begin{prb}
Suppose there are three cards. One card is red on both sides, one is green on both
sides, and one has a side of each color. Suppose one card is chosen at random (i.e.
blindly out of a bag). You are able to see that one side of the card is red. What is the
probability that the other side of the card is also red?
\end{prb}

\end{document}




