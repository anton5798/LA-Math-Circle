\documentclass[a4paper,12pt]{article} 
\usepackage{amsmath}
\usepackage[retainorgcmds]{IEEEtrantools}
\usepackage{graphicx}
\usepackage{amsfonts}
\usepackage{amsthm}
\usepackage{centernot}
\usepackage{setspace}
\usepackage{mathtools}
\usepackage{amssymb}
\usepackage{bm}
\usepackage[mathscr]{euscript}
\usepackage{pictexwd,dcpic}
\usepackage{tikz}
\usepackage{tikz-cd}
\usepackage[margin=1in]{geometry}
\usepackage{breqn}
\newtheoremstyle{perfect}% name
  {}%         Space above, empty = `usual value'
  {}%         Space below
  {}% Body font
  {}%         Indent amount (empty = no indent, \parindent = para indent)
  {\bfseries}% Thm head font
  {.}%        Punctuation after thm head
  {\newline}% Space after thm head: \newline = linebreak
  {}%         Thm head spec
\theoremstyle{perfect}
\newtheorem{lem}{Lemma}
\newtheorem{thm}{Theorem}
\newtheorem{dfn}{Definition}
\newtheorem{exm}{Example}
\newtheorem{prop}{Proposition}
\newtheorem{crl}{Corollary}
\newtheorem{Remark}{Remark}
\newtheorem{rem}{Reminder}
\newtheorem{prb}{Problem}
\newtheorem{exe}{Exercise}
\makeatletter
\newenvironment{cprb}[1]
  {\count@\c@prb
   \global\c@prb#1
    \global\advance\c@prb\m@ne
   \prb}
  {\endprb
   \global\c@prb\count@}
\makeatother
\makeatletter
\newenvironment{cexe}[1]
  {\count@\c@exe
   \global\c@exe#1
    \global\advance\c@exe\m@ne
   \exe}
  {\endexe
   \global\c@exe\count@}
\makeatother
\newcommand{\varline}{0}
\newcommand{\gen}[1]{\left< #1 \right>}
\newcommand{\ngen}[1]{\gen{\gen{#1}}}
\newcommand{\ov}[1]{\,\overline{#1}}
\DeclareMathOperator{\tor}{Tor}
\DeclareMathOperator{\aut}{Aut}
\DeclareMathOperator{\inn}{Inn}
\DeclareMathOperator{\im}{im}
\DeclareMathOperator{\imm}{Im}
\DeclareMathOperator{\ad}{ad}
\DeclareMathOperator{\Ad}{Ad}
\DeclareMathOperator{\Sp}{Sp}
\DeclareMathOperator{\SO}{SO}
\DeclareMathOperator{\SL}{SL}
\DeclareMathOperator{\SU}{SU}
\DeclareMathOperator{\GL}{GL}
\DeclareMathOperator{\PGL}{PGL}
\DeclareMathOperator{\re}{Re}
\DeclareMathOperator{\Hom}{Hom}
\DeclareMathOperator{\sym}{Sym}
\DeclareMathOperator{\ind}{Ind}
\DeclareMathOperator{\res}{Res}
\DeclareMathOperator{\sgn}{sgn}
\DeclareMathOperator{\End}{End}
\DeclareMathOperator{\colim}{colim}
\DeclareMathOperator{\coker}{coker}
\DeclareMathOperator{\Tr}{Tr}
\DeclareMathOperator{\intr}{int}
\DeclareMathOperator{\extr}{ext}
\DeclareMathOperator{\chr}{char}
\DeclareMathOperator{\supp}{supp}
\DeclareMathOperator{\hol}{Hol}
\DeclareMathOperator{\spec}{Spec}
\renewcommand{\Re}{\re}
\renewcommand{\Im}{\imm}
\newcommand{\eps}{\varepsilon}
\newcommand{\Mor}{\text{Mor}}
\newcommand{\cir}[1]{\mathrlap{\bigcirc}{\;#1}}
\newcommand{\Z}{\mathbb{Z}}
\newcommand{\Q}{\mathbb{Q}}
\newcommand{\R}{\mathbb{R}}
\newcommand{\C}{\mathbb{C}}
\newcommand{\F}{\mathbb{F}}
\newcommand{\N}{\mathbb{N}}
\newcommand{\lnorm}{\vartriangleleft}
\newcommand{\rnorm}{\vartriangleright}
\newcommand{\id}{\text{id}}
\newcommand{\dd}[1]{\mathrm{d}{#1}}
\newcommand{\p}[1]{\left( #1 \right)}
\newcommand{\parder}[2]{\frac{\partial #1}{\partial #2}}
\newcommand{\legendre}[2]{\left(\frac{#1}{#2}\right)}
\makeatletter
\renewcommand\part[1]{
\ifnum\pdfstrcmp{\varline}{1}=0
    \vspace{.10in}\textbf{\\#1)}
  \else
    \textbf{#1)}
  \fi\renewcommand{\varline}{1}}
\makeatother
\makeatletter
\newcommand{\tpmod}[1]{{\@displayfalse\pmod{#1}}}
\makeatother
\renewcommand{\restriction}{\mathord{\upharpoonright}}\author{Anton Lykov} 
\title{Lesson 3: Probability II }
\begin{document} 
%\setstretch{1}
\maketitle


\begin{dfn}[\textbf{Conditional Probability}]
Let $A, B \subset \Omega$ be two events, such that $\mathbf{P}(B) \neq 0$. 
The conditional probability of A given B, written $\mathbf{P}(A|B)$, 
is the probability of A given that B happened:
\begin{equation}\label{condprob}
\mathbf{P}(A|B) = \frac{\mathbf{P}(A \cap B)}{\mathbf{P}(B)}
\end{equation}

\noindent
In particular, if $\Omega$ is a finite sample space and every elementary outcome in $\Omega$ is equally likely, then 
\begin{equation}
\mathbf{P}(A|B) = \frac{|A \cap B|}{|B|}
\end{equation}
Recall that $|A|$ means the number of elements in $A$. 
\end{dfn}

\begin{thm}[\textbf{Law of Total Probability}]
Suppose that $A, B_1, \dots , B_n \subset \Omega$, with $B_1, \dots, B_n$ pairwise
disjoint (i.e. $B_i \cap B_j = \varnothing$ if $i \neq j$) and $B_1 \cup \dots \cup B_n = \Omega$. Then
\begin{equation}
\mathbf{P}(A) = \sum_{i=1}^n \mathbf{P}(A|B_i)\mathbf{P}(B_i)
\end{equation}
\end{thm}



\section{Introductory problems}

\begin{Remark}
In each problem below, describe how you are using conditional probability and/or the Law of Total Probability.
\end{Remark}

\begin{prb}
\textbf{a)} You roll a 6-sided die and someone (correctly) tells you that the roll was even. What would the probability of getting a 6 be (taking this information into account)? \\
\textbf{b)} Suppose you roll a 6-sided die twice. Given that the sum of the two numbers is $\geq 8$, what is the probability that the sum is 12? Compare this with the 
probability of getting a sum of 12 with no additional information given.
\end{prb}

\begin{prb}
Suppose you flip a fair coin 7 times. Given that the first 3 flips are
heads, what is the probability of getting exactly 4 heads total?
\end{prb}

\begin{prb}
Prove the Law of Total Probability.
\end{prb}

\begin{prb}
Two lines $l, k$ are perpendicular to each other and intersect at $O$. You roll a $10$-sided fair die and pick a point $A$ on $l$ such that $|AO|$ is equal to the number on the die. Then you roll the die again and similarly pick point $B$ on $k$. What is the probability that the area of $\triangle AOB$ is greater than $20$? 
\end{prb}



\section{Advanced problems}

\begin{prb}
Suppose you have three bags: bag A with 1 red and 3 green balls, bag B with 2R, 2G, and bag C
with 1R, 4G. You roll a die and then pick two balls from bag A if you roll $1- 3$, B if you
roll $4-5$, and C if you roll 6. Find the probability that both balls are the same color.
\end{prb}

\begin{prb}
Alex wants to celebrate her birthday with friends. Dad offers her a deal: ``There are $3$ days until your birthday. In each of these days, you play a chess game with me or mom, alternating opponents, one game per day. You can choose who to play with first. If you win $2$ games in a row -- you can celebrate with friends."
Who should Alex play first, if dad plays better than mom (i.e the probability of winning against dad is less)? 
\end{prb}

\begin{prb}[\textbf{Bayes' Rule}]
Let $A, B \subset \Omega$ be events such that $\mathbf{P}(B) > 0$. Prove that
\begin{equation}
\mathbf{P}(A | B) = \frac{ \mathbf{P}(B |A)\mathbf{P}(A)}{\mathbf{P}(B)}
\end{equation}
\end{prb}

\begin{prb}
A test for a certain rare disease is assumed to be correct 95\% of the time: if a person has the disease, the test results are positive with probability $0.95$, and if the person does not have the disease, the test results are negative with probability 0.95. The disease affects 1 out of every 1000 people in the population randomly. Given that the person just tested positive, what is the probability of actually having the disease? 
\end{prb}


 

\end{document}


