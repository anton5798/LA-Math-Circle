\documentclass[a4paper,12pt]{article} 
\usepackage{amsmath}
\usepackage[retainorgcmds]{IEEEtrantools}
\usepackage{graphicx}
\usepackage{amsfonts}
\usepackage{amsthm}
\usepackage{centernot}
\usepackage{setspace}
\usepackage{mathtools}
\usepackage{amssymb}
\usepackage{bm}
\usepackage[mathscr]{euscript}
\usepackage{pictexwd,dcpic}
\usepackage{tikz}
\usepackage{tikz-cd}
\usepackage[margin=1in]{geometry}
\usepackage{breqn}
\newtheoremstyle{perfect}% name
  {}%         Space above, empty = `usual value'
  {}%         Space below
  {}% Body font
  {}%         Indent amount (empty = no indent, \parindent = para indent)
  {\bfseries}% Thm head font
  {.}%        Punctuation after thm head
  {\newline}% Space after thm head: \newline = linebreak
  {}%         Thm head spec
\theoremstyle{perfect}
\newtheorem{lem}{Lemma}
\newtheorem{thm}{Theorem}
\newtheorem{dfn}{Definition}
\newtheorem{exm}{Example}
\newtheorem{prop}{Proposition}
\newtheorem{crl}{Corollary}
\newtheorem{Remark}{Remark}
\newtheorem{rem}{Reminder}
\newtheorem{prb}{Problem}
\newtheorem{exe}{Exercise}
\makeatletter
\newenvironment{cprb}[1]
  {\count@\c@prb
   \global\c@prb#1
    \global\advance\c@prb\m@ne
   \prb}
  {\endprb
   \global\c@prb\count@}
\makeatother
\makeatletter
\newenvironment{cexe}[1]
  {\count@\c@exe
   \global\c@exe#1
    \global\advance\c@exe\m@ne
   \exe}
  {\endexe
   \global\c@exe\count@}
\makeatother
\newcommand{\varline}{0}
\newcommand{\gen}[1]{\left< #1 \right>}
\newcommand{\ngen}[1]{\gen{\gen{#1}}}
\newcommand{\ov}[1]{\,\overline{#1}}
\DeclareMathOperator{\tor}{Tor}
\DeclareMathOperator{\aut}{Aut}
\DeclareMathOperator{\inn}{Inn}
\DeclareMathOperator{\im}{im}
\DeclareMathOperator{\imm}{Im}
\DeclareMathOperator{\ad}{ad}
\DeclareMathOperator{\Ad}{Ad}
\DeclareMathOperator{\Sp}{Sp}
\DeclareMathOperator{\SO}{SO}
\DeclareMathOperator{\SL}{SL}
\DeclareMathOperator{\SU}{SU}
\DeclareMathOperator{\GL}{GL}
\DeclareMathOperator{\PGL}{PGL}
\DeclareMathOperator{\re}{Re}
\DeclareMathOperator{\Hom}{Hom}
\DeclareMathOperator{\sym}{Sym}
\DeclareMathOperator{\ind}{Ind}
\DeclareMathOperator{\res}{Res}
\DeclareMathOperator{\sgn}{sgn}
\DeclareMathOperator{\End}{End}
\DeclareMathOperator{\colim}{colim}
\DeclareMathOperator{\coker}{coker}
\DeclareMathOperator{\Tr}{Tr}
\DeclareMathOperator{\intr}{int}
\DeclareMathOperator{\extr}{ext}
\DeclareMathOperator{\chr}{char}
\DeclareMathOperator{\supp}{supp}
\DeclareMathOperator{\hol}{Hol}
\DeclareMathOperator{\spec}{Spec}
\renewcommand{\Re}{\re}
\renewcommand{\Im}{\imm}
\newcommand{\eps}{\varepsilon}
\newcommand{\Mor}{\text{Mor}}
\newcommand{\cir}[1]{\mathrlap{\bigcirc}{\;#1}}
\newcommand{\Z}{\mathbb{Z}}
\newcommand{\Q}{\mathbb{Q}}
\newcommand{\R}{\mathbb{R}}
\newcommand{\C}{\mathbb{C}}
\newcommand{\EE}{\mathbb{E}}
\newcommand{\F}{\mathbb{F}}
\newcommand{\N}{\mathbb{N}}
\newcommand{\PP}{\mathbf{P}}
\newcommand{\lnorm}{\vartriangleleft}
\newcommand{\rnorm}{\vartriangleright}
\newcommand{\id}{\text{id}}
\newcommand{\dd}[1]{\mathrm{d}{#1}}
\newcommand{\p}[1]{\left( #1 \right)}
\newcommand{\parder}[2]{\frac{\partial #1}{\partial #2}}
\newcommand{\legendre}[2]{\left(\frac{#1}{#2}\right)}
\makeatletter
\renewcommand\part[1]{
\ifnum\pdfstrcmp{\varline}{1}=0
    \vspace{.10in}\textbf{\\#1)}
  \else
    \textbf{#1)}
  \fi\renewcommand{\varline}{1}}
\makeatother
\makeatletter
\newcommand{\tpmod}[1]{{\@displayfalse\pmod{#1}}}
\makeatother
\renewcommand{\restriction}{\mathord{\upharpoonright}}\author{Anton Lykov} 
\title{Lesson 4: Probability III}
\begin{document} 
%\setstretch{1}
\maketitle


\begin{dfn}[\textbf{Random Variable - informal}]
A random variable is a quantifiable experiment. That
is, an experiment with numerical outcomes. 
Random variables are usually denoted by capital letters (X, Y, Z, etc.)
For example, we could define random variable $X$ = the number of heads
after tossing $5$ fair coins.
\end{dfn}


\begin{dfn}[\textbf{Random Variable - formal}]
Let $\Omega$ be a sample space, and $\PP$ be a Probability Function on $\Omega$. A \textbf{random variable} X is a function $X : \Omega \to \R$. That is, $X$ assigns a real number to each elementary outcome.
\end{dfn}


\begin{dfn}[\textbf{Probability Mass Function}]
The \textbf{probability mass function} of X, denoted $p_X : \R \to [0, 1]$ is defined by:
$$p_X(x) = \PP(X = x) = \PP(\{X = x\}) = \PP(\{\omega \in \Omega: X(\omega) = x\})$$
\end{dfn}

\begin{dfn}[\textbf{Expected Value}]
The \textbf{expected value} of a random variable $X$ is defined as:
$$\sum_{x \in \R} x p_X(x)$$

\end{dfn}


\section{Introductory problems}

\begin{prb}
Let X be a discrete random variable on a finite sample space $\Omega$. Prove that $\sum_{x \in \R} p_X(x) = 1$.
\end{prb}

\begin{prb}
You roll an unfair seven-sided die with faces 1, 2, ..., 7, where the probability of landing on each side is proportional to the value on the side. That is, you are twice as likely to roll a 2 as you are to roll a 1, and 7/3 more likely to roll a 7 than a 3. Let X be the value of the face that the die lands on. Find the probability mass function of X and the expected value of X.
\end{prb}

\begin{prb}
Suppose your friend offers you the following deal: a six-sided die is rolled and you get to wager any number of jolly ranchers on the result being even or odd. If you are correct, you win as many jolly ranchers as you wagered. If you are incorrect, you lose what you wagered. 
Assuming you just went trick-or-treating and have an infinite supply of jolly ranchers, come up with a strategy with a positive expected value of jolly ranchers won and show that it works. 
\end{prb}

\begin{prb}
You are given a ten sided die with values 1-10, and you are allowed to roll it either once or twice.
After the first dice roll you note the score and have to make a decision whether you want to try a second roll or not.
In case you decided to roll a second time, you add the values from both dices to get your final score.
If your score is less than or equal to 13 you will receive the equivalent amount in pounds as a payout. However if your score exceeds 13 you won't receive a payout.
What would be the optimal strategy? \\
\textbf{Hint:} What is the expected value in each case?
\end{prb}



\section{Advanced problems}

\begin{prb}
Suppose you flip a biased coin that lands heads with probability $p$. Let $X$ be the number of heads in $3$ throws. Find the probability mass function of $X$, $p_X$, as well as the expected value of X, $\EE(X)$.
\end{prb}
 
 \begin{prb} 
\textbf{a)} Suppose you're on a game show, and you're given the choice of three doors. Behind one door there is a car, behind the other - goats. You pick a door, say 1, and the host, who knows what's behind the doors, opens another door, say 3, which has a goat. He then says to you, ``Do you now want to pick door 2 instead?" Is it to your advantage to switch your choice (assuming you don't want a goat)? What is the probability of winning the car if you switch vs. if you stick with door 1? \\
\noindent
\textbf{b)} Now suppose that two of the three doors contain a car. Then after picking a door, say 1, the host, who knows what's behind the doors, opens another door, say 3, which has a car. Is it advantageous to switch doors or stick with your original pick?
 \end{prb}

\begin{prb}
\textbf{a)} You are flipping a fair coin until it lands heads. After it lands heads, you stop. Let $X$ be the number of flips you make. Find the probability mass function of $X$. 
\newline
\textbf{b)} Find the expected value of X.  \\
\textbf{c)} Find the expected value of X if the coin lands tails with probability $p$.
\end{prb}

\begin{prb}
Two players, A and B, have 4 fair coins each. They both flip their four coins simultaneously, and if the number of heads among 4 coins is the same for both A and B, B pays 4 dollars to A, otherwise A pays 1 dollar to B. Who would you prefer to be - player A or player B?
\end{prb}


\end{document}


