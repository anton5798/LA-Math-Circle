\documentclass[a4paper,12pt]{article} 
\usepackage{amsmath}
\usepackage[retainorgcmds]{IEEEtrantools}
\usepackage{graphicx}
\usepackage{amsfonts}
\usepackage{amsthm}
\usepackage{centernot}
\usepackage{setspace}
\usepackage{mathtools}
\usepackage{amssymb}
\usepackage{bm}
\usepackage[mathscr]{euscript}
\usepackage{pictexwd,dcpic}
\usepackage{tikz}
\usepackage{tikz-cd}
\usepackage[margin=1in]{geometry}
\usepackage{breqn}
\newtheoremstyle{perfect}% name
  {}%         Space above, empty = `usual value'
  {}%         Space below
  {}% Body font
  {}%         Indent amount (empty = no indent, \parindent = para indent)
  {\bfseries}% Thm head font
  {.}%        Punctuation after thm head
  {\newline}% Space after thm head: \newline = linebreak
  {}%         Thm head spec
\theoremstyle{perfect}
\newtheorem{lem}{Lemma}
\newtheorem{thm}{Theorem}
\newtheorem{dfn}{Definition}
\newtheorem{exm}{Example}
\newtheorem{prop}{Proposition}
\newtheorem{crl}{Corollary}
\newtheorem{Remark}{Remark}
\newtheorem{rem}{Reminder}
\newtheorem{prb}{Problem}
\newtheorem{exe}{Exercise}
\makeatletter
\newenvironment{cprb}[1]
  {\count@\c@prb
   \global\c@prb#1
    \global\advance\c@prb\m@ne
   \prb}
  {\endprb
   \global\c@prb\count@}
\makeatother
\makeatletter
\newenvironment{cexe}[1]
  {\count@\c@exe
   \global\c@exe#1
    \global\advance\c@exe\m@ne
   \exe}
  {\endexe
   \global\c@exe\count@}
\makeatother
\newcommand{\varline}{0}
\newcommand{\gen}[1]{\left< #1 \right>}
\newcommand{\ngen}[1]{\gen{\gen{#1}}}
\newcommand{\ov}[1]{\,\overline{#1}}
\DeclareMathOperator{\tor}{Tor}
\DeclareMathOperator{\aut}{Aut}
\DeclareMathOperator{\inn}{Inn}
\DeclareMathOperator{\im}{im}
\DeclareMathOperator{\imm}{Im}
\DeclareMathOperator{\ad}{ad}
\DeclareMathOperator{\Ad}{Ad}
\DeclareMathOperator{\Sp}{Sp}
\DeclareMathOperator{\SO}{SO}
\DeclareMathOperator{\SL}{SL}
\DeclareMathOperator{\SU}{SU}
\DeclareMathOperator{\GL}{GL}
\DeclareMathOperator{\PGL}{PGL}
\DeclareMathOperator{\re}{Re}
\DeclareMathOperator{\Hom}{Hom}
\DeclareMathOperator{\sym}{Sym}
\DeclareMathOperator{\ind}{Ind}
\DeclareMathOperator{\res}{Res}
\DeclareMathOperator{\sgn}{sgn}
\DeclareMathOperator{\End}{End}
\DeclareMathOperator{\colim}{colim}
\DeclareMathOperator{\coker}{coker}
\DeclareMathOperator{\Tr}{Tr}
\DeclareMathOperator{\intr}{int}
\DeclareMathOperator{\extr}{ext}
\DeclareMathOperator{\chr}{char}
\DeclareMathOperator{\supp}{supp}
\DeclareMathOperator{\hol}{Hol}
\DeclareMathOperator{\spec}{Spec}
\renewcommand{\Re}{\re}
\renewcommand{\Im}{\imm}
\newcommand{\eps}{\varepsilon}
\newcommand{\Mor}{\text{Mor}}
\newcommand{\cir}[1]{\mathrlap{\bigcirc}{\;#1}}
\newcommand{\Z}{\mathbb{Z}}
\newcommand{\Q}{\mathbb{Q}}
\newcommand{\R}{\mathbb{R}}
\newcommand{\C}{\mathbb{C}}
\newcommand{\F}{\mathbb{F}}
\newcommand{\N}{\mathbb{N}}
\newcommand{\lnorm}{\vartriangleleft}
\newcommand{\rnorm}{\vartriangleright}
\newcommand{\id}{\text{id}}
\newcommand{\dd}[1]{\mathrm{d}{#1}}
\newcommand{\p}[1]{\left( #1 \right)}
\newcommand{\parder}[2]{\frac{\partial #1}{\partial #2}}
\newcommand{\legendre}[2]{\left(\frac{#1}{#2}\right)}
\makeatletter
\renewcommand\part[1]{
\ifnum\pdfstrcmp{\varline}{1}=0
    \vspace{.10in}\textbf{\\#1)}
  \else
    \textbf{#1)}
  \fi\renewcommand{\varline}{1}}
\makeatother
\makeatletter
\newcommand{\tpmod}[1]{{\@displayfalse\pmod{#1}}}
\makeatother
\renewcommand{\restriction}{\mathord{\upharpoonright}}\author{Anton Lykov} 
\title{Lesson 2: Set theory and Probability }
\begin{document} 
%\setstretch{1}
\maketitle

\begin{dfn}
The \textbf{union} of two sets $A$ and $B$ is a set consisting of all elements that belong to either $A$ or $B$, and is denoted by $A \cup B$.\\
\\
The \textbf{intersection} of two sets $A$ and $B$ is a set consisting of all elements that belong to both $A$ and $B$, and is denoted by $A \cap B$. \\ 
\\
The \textbf{difference} of two sets $A$ and $B$ is a set consisting of all elements that belong to $A$, but not to $B$, and is denoted by $A \backslash B$. \\
\\
The \textbf{complement} of a set $A$ with respect to some universal set $U$ is the difference $U \backslash A$, denoted by $A^c$. In this handout, the universal set will be the sample space (see Problem 5).
\end{dfn}

\begin{prb}
Let $A = \{57, 91, 179, 239\}, B = \{91, 239, 2014\}, C = \{2, 57, 239, 2014\},
D = \{2, 91, 2014, 2017\}$ and $E$ is the set of all flying crocodiles. 
Draw the Venn diagram and find the following sets:
\begin{enumerate}
\item  $A \cap B$
\item $(A \cap B) \cup D$
\item $C \cap (D \cap B)$
\item $(A \cup B) \cap (C \cup D)$
\item $(A \cap B) \cup (C \cap D)$ 
\item $(A \cap (B \cap C)) \cap D$
\item $(A \cup (B \cap C)) \cap D$
\item $(C \cap A) \cup ((A \cup (C \cap D)) \cap B)$
\item Is $A \subset B$ true?
\item Is $E \subset A$ true?
\end{enumerate}

\end{prb}

\begin{prb}
Calculate the probability of flipping a fair coin three times and getting 2H and 1T 
\begin{enumerate}
\item In this order
\item In any order
\end{enumerate}
\end{prb} 

\begin{prb}
Calculate the probability of randomly rearranging tiles with the letters of FORMULA and getting a word starting with a vowel.
\end{prb} 

\begin{prb}
Calculate the probability of rolling a six-sided die three times and getting a sum of three outcomes equal to 10.
\end{prb} 


\begin{prb}
Let $\Omega = \{\omega_1, \omega_2, ..., \omega_n\}$ be a \textbf{sample space}. Each $\omega_i$ is called an \textbf{elementary (atomic) event}. Recall from lecture the definition of \textbf{Probability function}. Given a set $A \subset \Omega$, the probability function assigns a number $P(A)$ to the set A. A set $A \subset \Omega$ is called an \textbf{event}. The number $P(A)$ denotes the probability that the event A will occur. The probability law satisfies the axioms below:
\begin{enumerate}
\item $P(A) \geq 0$ for any event $A \subset \Omega$.
\item $P(A \cup B) = P(A) + P(B)$, if $A$ and $B$ are disjoint events (i.e. $A \cap B = \varnothing$).
\item $P(\Omega) = 1$.
\end{enumerate}

\noindent
Recall from lecture that $A^c = \Omega \backslash A$. For any events $A, B \subset \Omega$, prove the following:
\begin{enumerate}
\item $P(\varnothing) = 0$
\item $0 \leq P(A) \leq 1$
\item $P(A \cup B) = P(A) + P(B) - P(A \cap B)$
\item $P(A^c) = 1 - P(A)$
\end{enumerate}
\end{prb}

\begin{exm}
Suppose you are flip a fair coin twice. In this experiment, the sample space looks like this: 
\{TT, TH, HT, HH\}. The sample space represents all possible outcomes of your experiments. The probability of each elementary event is $\frac{1}{4}$. We can also look at non-elementary events. For example, let $A$ be an event that there is at least one H. Then A = \{TH, HT, HH\}, and $P(A) = \frac{3}{4}$. 
\end{exm}

\begin{Remark}
In problems 6 and 7, describe the sample space and probability function using the definitions above. Also, describe events you are using in your solution.
\end{Remark}

\begin{prb}
In a bucket, there are $m$ white and $(n-m)$ black balls. Each turn, Michael and Peter take one ball from the bucket, Michael starts. The one who takes the white ball first wins. Find the probability of Peter winning this game if:
\begin{enumerate}
\item $n=5, m=1$
\item $n=7, m=2$
\item For any natural numbers $n,m$, find the desired probability as a formula of $n,m$.
\end{enumerate}
\end{prb}


\begin{prb}
Suppose that birthdays are equally distributed between all months of the year.
\begin{enumerate}
\item Find the probability that in a group of  $n$ people, two share the same birthday month.
\item Find the smallest $n$ so that this probability is $\geq 50\%$.
\end{enumerate}
\end{prb} 

\begin{prb}
You have two congruent cardboard triangles, and one of their angles is equal to $\alpha$. 
Place them on the plane in such a way, that three vertices form an angle equal to $\alpha / 2$. Using any instruments, including pencils, is not allowed!
\end{prb}

 




\end{document}


