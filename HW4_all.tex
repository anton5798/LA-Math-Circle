\documentclass[a4paper,12pt]{article} 
\usepackage{amsmath}
\usepackage[retainorgcmds]{IEEEtrantools}
\usepackage{graphicx}
\usepackage{amsfonts}
\usepackage{amsthm}
\usepackage{centernot}
\usepackage{setspace}
\usepackage{mathtools}
\usepackage{amssymb}
\usepackage{bm}
\usepackage[mathscr]{euscript}
\usepackage{pictexwd,dcpic}
\usepackage{tikz}
\usepackage{tikz-cd}
\usepackage[margin=1in]{geometry}
\usepackage{breqn}
\newtheoremstyle{perfect}% name
  {}%         Space above, empty = `usual value'
  {}%         Space below
  {}% Body font
  {}%         Indent amount (empty = no indent, \parindent = para indent)
  {\bfseries}% Thm head font
  {.}%        Punctuation after thm head
  {\newline}% Space after thm head: \newline = linebreak
  {}%         Thm head spec
\theoremstyle{perfect}
\newtheorem{lem}{Lemma}
\newtheorem{thm}{Theorem}
\newtheorem{dfn}{Definition}
\newtheorem{exm}{Example}
\newtheorem{prop}{Proposition}
\newtheorem{crl}{Corollary}
\newtheorem{sol}{Solution}
\newtheorem{rem}{Reminder}
\newtheorem{prb}{Problem}
\newtheorem{exe}{Exercise}
\makeatletter
\newenvironment{cprb}[1]
  {\count@\c@prb
   \global\c@prb#1
    \global\advance\c@prb\m@ne
   \prb}
  {\endprb
   \global\c@prb\count@}
\makeatother
\makeatletter
\newenvironment{cexe}[1]
  {\count@\c@exe
   \global\c@exe#1
    \global\advance\c@exe\m@ne
   \exe}
  {\endexe
   \global\c@exe\count@}
\makeatother
\newcommand{\varline}{0}
\newcommand{\gen}[1]{\left< #1 \right>}
\newcommand{\ngen}[1]{\gen{\gen{#1}}}
\newcommand{\ov}[1]{\,\overline{#1}}
\DeclareMathOperator{\tor}{Tor}
\DeclareMathOperator{\aut}{Aut}
\DeclareMathOperator{\inn}{Inn}
\DeclareMathOperator{\im}{im}
\DeclareMathOperator{\imm}{Im}
\DeclareMathOperator{\ad}{ad}
\DeclareMathOperator{\Ad}{Ad}
\DeclareMathOperator{\Sp}{Sp}
\DeclareMathOperator{\SO}{SO}
\DeclareMathOperator{\SL}{SL}
\DeclareMathOperator{\SU}{SU}
\DeclareMathOperator{\GL}{GL}
\DeclareMathOperator{\PGL}{PGL}
\DeclareMathOperator{\re}{Re}
\DeclareMathOperator{\Hom}{Hom}
\DeclareMathOperator{\sym}{Sym}
\DeclareMathOperator{\ind}{Ind}
\DeclareMathOperator{\res}{Res}
\DeclareMathOperator{\sgn}{sgn}
\DeclareMathOperator{\End}{End}
\DeclareMathOperator{\colim}{colim}
\DeclareMathOperator{\coker}{coker}
\DeclareMathOperator{\Tr}{Tr}
\DeclareMathOperator{\intr}{int}
\DeclareMathOperator{\extr}{ext}
\DeclareMathOperator{\chr}{char}
\DeclareMathOperator{\supp}{supp}
\DeclareMathOperator{\hol}{Hol}
\DeclareMathOperator{\spec}{Spec}
\renewcommand{\Re}{\re}
\renewcommand{\Im}{\imm}
\newcommand{\eps}{\varepsilon}
\newcommand{\Mor}{\text{Mor}}
\newcommand{\cir}[1]{\mathrlap{\bigcirc}{\;#1}}
\newcommand{\Z}{\mathbb{Z}}
\newcommand{\Q}{\mathbb{Q}}
\newcommand{\R}{\mathbb{R}}
\newcommand{\C}{\mathbb{C}}
\newcommand{\F}{\mathbb{F}}
\newcommand{\N}{\mathbb{N}}
\newcommand{\lnorm}{\vartriangleleft}
\newcommand{\rnorm}{\vartriangleright}
\newcommand{\id}{\text{id}}
\newcommand{\dd}[1]{\mathrm{d}{#1}}
\newcommand{\p}[1]{\left( #1 \right)}
\newcommand{\parder}[2]{\frac{\partial #1}{\partial #2}}
\newcommand{\legendre}[2]{\left(\frac{#1}{#2}\right)}
\makeatletter
\renewcommand\part[1]{
\ifnum\pdfstrcmp{\varline}{1}=0
    \vspace{.10in}\textbf{\\#1)}
  \else
    \textbf{#1)}
  \fi\renewcommand{\varline}{1}}
\makeatother
\makeatletter
\newcommand{\tpmod}[1]{{\@displayfalse\pmod{#1}}}
\makeatother
\renewcommand{\restriction}{\mathord{\upharpoonright}}\author{Konstantin Miagkov} 
\title{Homework 4: Quadratic Inequalities}
\begin{document} 
%\setstretch{1}
\maketitle

\section{Reading}

\begin{sol}[L3.2a]
Let us go through the functions:\\
\part{1} $f(x) = x\cdot \lvert x \rvert$. We will show that this is odd, using the central property of the absolute value $\lvert x \rvert = \lvert -x \rvert$: $$f(-x) = -x\cdot \lvert -x \rvert = -x\cdot \lvert x \rvert = -f(x)$$
\part{2} $f(x) = \lvert x+1\rvert - \lvert x-1\rvert$. This is also odd: $$f(-x) = \lvert -x+1\rvert - \lvert -x-1\rvert = \lvert x - 1\rvert - \lvert x + 1\rvert = -f(x)$$
\part{3} $f(x) = \lvert x+1\rvert + \lvert x-1\rvert$. This is even: $$f(-x) = \lvert -x+1\rvert + \lvert -x-1\rvert = \lvert x-1\rvert + \lvert x+1\rvert = f(x)$$
\part{4} $f(x) = 3x-x^2$. This is neither odd nor even. Indeed, $f(1) = 2$ and $f(-1) = -4$, which means that $f(-x) \neq f(x)$ and $f(-x) \neq -f(x)$ at least at $x = 1$.
\end{sol}

\begin{sol}[H3.2]
Let $I$ be the intersection of diagonals of $ABCD$, which is incidentally also the center of the inscribed circle. Let $E$ be the point at which the inscribed circle is tangent to $AB$, and $F$ -- the point of tangency to $AD$. Then note that $AE = AF$ and $IE = IF$. Then $\triangle AEI = \triangle AFI$, which in turn implies $\angle BAI = \angle DAI$. Similarly, we can show that $\angle BCI = \angle DCI$. The two angle equalities imply that $\triangle ABC = \triangle ADC$, which means that $AB = AD$ and $CD = BC$. Similarly we cab get that $\triangle BAD = \triangle BCD$, from where $AD = AB = BC$ and $$

Now we can remember that $AB + CD = AD + BC$, and we get $AB = CD = AD = BC$ which means that $ABCD$ is a rhombus.
\end{sol}

\section{Homework}

\begin{prb}
Show that out of all rectangles with a fixed perimeter, the square has the largest area.
\end{prb}

\begin{prb}
The circle with center $O$ is inscribed in the quadrilateral $ABCD$. Show that $\angle AOB + \angle COD = 180^\circ$.
\end{prb}

\end{document}