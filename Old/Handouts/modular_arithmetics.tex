\documentclass[a4paper,12pt]{article} 
\usepackage{amsmath}
\usepackage[retainorgcmds]{IEEEtrantools}
\usepackage{graphicx}
\usepackage{amsfonts}
\usepackage{amsthm}
\usepackage{centernot}
\usepackage{setspace}
\usepackage{mathtools}
\usepackage{amssymb}
\usepackage{bm}
\usepackage[mathscr]{euscript}
\usepackage{pictexwd,dcpic}
\usepackage{tikz}
\usepackage{tikz-cd}
\usepackage[margin=1in]{geometry}
\usepackage{breqn}
\newtheoremstyle{perfect}% name
  {}%         Space above, empty = `usual value'
  {}%         Space below
  {}% Body font
  {}%         Indent amount (empty = no indent, \parindent = para indent)
  {\bfseries}% Thm head font
  {.}%        Punctuation after thm head
  {\newline}% Space after thm head: \newline = linebreak
  {}%         Thm head spec
\theoremstyle{perfect}
\newtheorem{lem}{Lemma}
\newtheorem{thm}{Theorem}
\newtheorem{dfn}{Definition}
\newtheorem{exm}{Example}
\newtheorem{prop}{Proposition}
\newtheorem{crl}{Corollary}
\newtheorem{rem}{Reminder}
\newtheorem{prb}{Problem}
\newtheorem{sol}{Solution}
\newtheorem{exe}{Exercise}
\makeatletter
\newenvironment{cprb}[1]
  {\count@\c@prb
   \global\c@prb#1
    \global\advance\c@prb\m@ne
   \prb}
  {\endprb
   \global\c@prb\count@}
\makeatother
\makeatletter
\newenvironment{cexe}[1]
  {\count@\c@exe
   \global\c@exe#1
    \global\advance\c@exe\m@ne
   \exe}
  {\endexe
   \global\c@exe\count@}
\makeatother
\newcommand{\varline}{0}
\newcommand{\gen}[1]{\left< #1 \right>}
\newcommand{\ngen}[1]{\gen{\gen{#1}}}
\newcommand{\ov}[1]{\,\overline{#1}}
\DeclareMathOperator{\tor}{Tor}
\DeclareMathOperator{\aut}{Aut}
\DeclareMathOperator{\inn}{Inn}
\DeclareMathOperator{\im}{im}
\DeclareMathOperator{\imm}{Im}
\DeclareMathOperator{\ad}{ad}
\DeclareMathOperator{\Ad}{Ad}
\DeclareMathOperator{\Sp}{Sp}
\DeclareMathOperator{\SO}{SO}
\DeclareMathOperator{\SL}{SL}
\DeclareMathOperator{\SU}{SU}
\DeclareMathOperator{\GL}{GL}
\DeclareMathOperator{\PGL}{PGL}
\DeclareMathOperator{\re}{Re}
\DeclareMathOperator{\Hom}{Hom}
\DeclareMathOperator{\sym}{Sym}
\DeclareMathOperator{\ind}{Ind}
\DeclareMathOperator{\res}{Res}
\DeclareMathOperator{\sgn}{sgn}
\DeclareMathOperator{\End}{End}
\DeclareMathOperator{\colim}{colim}
\DeclareMathOperator{\coker}{coker}
\DeclareMathOperator{\Tr}{Tr}
\DeclareMathOperator{\intr}{int}
\DeclareMathOperator{\extr}{ext}
\DeclareMathOperator{\chr}{char}
\DeclareMathOperator{\supp}{supp}
\DeclareMathOperator{\hol}{Hol}
\DeclareMathOperator{\spec}{Spec}
\renewcommand{\Re}{\re}
\renewcommand{\Im}{\imm}
\newcommand{\eps}{\varepsilon}
\newcommand{\Mor}{\text{Mor}}
\newcommand{\cir}[1]{\mathrlap{\bigcirc}{\;#1}}
\newcommand{\Z}{\mathbb{Z}}
\newcommand{\Q}{\mathbb{Q}}
\newcommand{\R}{\mathbb{R}}
\newcommand{\C}{\mathbb{C}}
\newcommand{\F}{\mathbb{F}}
\newcommand{\N}{\mathbb{N}}
\newcommand{\lnorm}{\vartriangleleft}
\newcommand{\rnorm}{\vartriangleright}
\newcommand{\id}{\text{id}}
\newcommand{\dd}[1]{\mathrm{d}{#1}}
\newcommand{\p}[1]{\left( #1 \right)}
\newcommand{\parder}[2]{\frac{\partial #1}{\partial #2}}
\newcommand{\legendre}[2]{\left(\frac{#1}{#2}\right)}
\makeatletter
\renewcommand\part[1]{
\ifnum\pdfstrcmp{\varline}{1}=0
    \vspace{.10in}\textbf{\\#1)}
  \else
    \textbf{#1)}
  \fi\renewcommand{\varline}{1}}
\makeatother
\makeatletter
\newcommand{\tpmod}[1]{{\@displayfalse\pmod{#1}}}
\makeatother
\renewcommand{\restriction}{\mathord{\upharpoonright}}\author{Anton Lykov} 
\title{Lesson 5, problem 3. Divisibility and remainders}
\begin{document} 
%\setstretch{1}
\maketitle

\begin{prb}
The number $8^{2019}$ is written on the board. At each step it is replaced by the sum of its digits, until a 1-digit number is left. What is that one-digit number?
\end{prb}

\begin{sol}
Let's first prove a lemma:
\begin{lem}[Divisibility rule by $9$]
Any nonnegative integer $A$ has the same remainder modulo $9$ (i.e., has the same remainder after dividing by $9$) as does the sum of its digits. 
\end{lem}
\begin{proof}
Let's write $A = \overline{a_na_{n-1}...a_1a_0}$, where bar denotes that $a_n,...,a_0$ are digits in $A$. For example, if $a_2 = 9, a_1 = 3, a_0 = 7$, then $\overline{a_2a_1a_0} = 937$. Note that 
$$A = 10^na_n + 10^{n-1}a_{n-1}+ ... + 10^1a_1 + 10^0a_0$$
Looking at the difference $A - (a_n + a_{n-1} + ... + a_0)$, we see that the difference is divisible by $9$:
\begin{align*}
A - (a_n  +  a_{n-1} + ... + a_0) &= 10^na_n + 10^{n-1}a_{n-1}+ ... + 10^1a_1 + 10^0a_0 \\ 
& - ( a_n +  a_{n-1} + ... +  a_1 +  a_0) \\
&=  \underbrace{99...9}_{n-1} a_n + \underbrace{99...9}_{n-2} a_{n-1} + ... + 99a_2 + 9a_1\\
&= 9*(\underbrace{11...1}_{n-1} a_n + \underbrace{11...1}_{n-2} a_{n-1} + ... + 11a_2 + a_1)
\end{align*} 
If $A$ and $(a_n  +  a_{n-1} + ... + a_0)$ had different remainders modulo $9$, their difference would have nonzero remainder modulo $9$, and thus would not be divisible by $9$. Therefore, $A$ and $(a_n  +  a_{n-1} + ... + a_0)$ have the same remainder modulo $9$. 
\end{proof}

\noindent
Note that, in particular, this means that if the sum of the digits of $A$ is divisible by $9$ (remainder is $0$), then $A$ itself is divisible by $9$. This statement is probably more familiar to you! \\

\noindent
Let's return to the original problem. We can see an invariant:
\textbf{The remainder of our number modulo 9}. Indeed, we just proved that the remainder stays the same when a number is substituted by the sum of its digits! 
This means that if we find the remainder modulo $9$ of the original number $8^{2019}$, the remainder of the one-digit number left on the board would be the same.

To finish the proof, we'll need one more thing. Suppose two integers $A$ and $B$ have remainders $a$ and $b$ modulo $9$. Then the product $AB$ has the same remainder modulo $9$ as $ab$. Indeed, we can write:
$$AB = (9k + a)(9n +b) = \underbrace{81kn + 9kb + 9an}_{divisible \; by \; 9} + ab$$


We now need to find the remainder of $8^{2019}$ modulo 9.
$8$ has remainder $8$ modulo $9$ (obviously). $8^2$ has remainder $1$ (check!). Then, by the fact above, $8^3$ has remainder $8$ modulo $9$. Similarly, $8^4$ has remainder $8$ and so on. We can see that remainder of $8^{n}$ will be $1$ if $n$ is even, and $8$ if $n$ is odd. Thus, $8^{2019}$ has remainder $8$, and so the remaining one-digit number's remainder is also $8$. The only such number is $8$.
\end{sol}
	




\end{document}