\documentclass[a4paper,12pt]{article} 
\usepackage{amsmath}
\usepackage[retainorgcmds]{IEEEtrantools}
\usepackage{graphicx}
\usepackage{amsfonts}
\usepackage{amsthm}
\usepackage{centernot}
\usepackage{setspace}
\usepackage{mathtools}
\usepackage{amssymb}
\usepackage{bm}
\usepackage[mathscr]{euscript}
\usepackage{pictexwd,dcpic}
\usepackage{tikz}
\usepackage{tikz-cd}
\usepackage[margin=1in]{geometry}
\usepackage{breqn}
\newtheoremstyle{perfect}% name
  {}%         Space above, empty = `usual value'
  {}%         Space below
  {}% Body font
  {}%         Indent amount (empty = no indent, \parindent = para indent)
  {\bfseries}% Thm head font
  {.}%        Punctuation after thm head
  {\newline}% Space after thm head: \newline = linebreak
  {}%         Thm head spec
\theoremstyle{perfect}
\newtheorem{lem}{Lemma}
\newtheorem{thm}{Theorem}
\newtheorem{dfn}{Definition}
\newtheorem{exm}{Example}
\newtheorem{prop}{Proposition}
\newtheorem{crl}{Corollary}
\newtheorem{rem}{Reminder}
\newtheorem{prb}{Problem}
\newtheorem{sol}{Solution}
\newtheorem{exe}{Exercise}
\makeatletter
\newenvironment{cprb}[1]
  {\count@\c@prb
   \global\c@prb#1
    \global\advance\c@prb\m@ne
   \prb}
  {\endprb
   \global\c@prb\count@}
\makeatother
\makeatletter
\newenvironment{cexe}[1]
  {\count@\c@exe
   \global\c@exe#1
    \global\advance\c@exe\m@ne
   \exe}
  {\endexe
   \global\c@exe\count@}
\makeatother
\newcommand{\varline}{0}
\newcommand{\gen}[1]{\left< #1 \right>}
\newcommand{\ngen}[1]{\gen{\gen{#1}}}
\newcommand{\ov}[1]{\,\overline{#1}}
\DeclareMathOperator{\tor}{Tor}
\DeclareMathOperator{\aut}{Aut}
\DeclareMathOperator{\inn}{Inn}
\DeclareMathOperator{\im}{im}
\DeclareMathOperator{\imm}{Im}
\DeclareMathOperator{\ad}{ad}
\DeclareMathOperator{\Ad}{Ad}
\DeclareMathOperator{\Sp}{Sp}
\DeclareMathOperator{\SO}{SO}
\DeclareMathOperator{\SL}{SL}
\DeclareMathOperator{\SU}{SU}
\DeclareMathOperator{\GL}{GL}
\DeclareMathOperator{\PGL}{PGL}
\DeclareMathOperator{\re}{Re}
\DeclareMathOperator{\Hom}{Hom}
\DeclareMathOperator{\sym}{Sym}
\DeclareMathOperator{\ind}{Ind}
\DeclareMathOperator{\res}{Res}
\DeclareMathOperator{\sgn}{sgn}
\DeclareMathOperator{\End}{End}
\DeclareMathOperator{\colim}{colim}
\DeclareMathOperator{\coker}{coker}
\DeclareMathOperator{\Tr}{Tr}
\DeclareMathOperator{\intr}{int}
\DeclareMathOperator{\extr}{ext}
\DeclareMathOperator{\chr}{char}
\DeclareMathOperator{\supp}{supp}
\DeclareMathOperator{\hol}{Hol}
\DeclareMathOperator{\spec}{Spec}
\renewcommand{\Re}{\re}
\renewcommand{\Im}{\imm}
\newcommand{\eps}{\varepsilon}
\newcommand{\Mor}{\text{Mor}}
\newcommand{\cir}[1]{\mathrlap{\bigcirc}{\;#1}}
\newcommand{\Z}{\mathbb{Z}}
\newcommand{\Q}{\mathbb{Q}}
\newcommand{\R}{\mathbb{R}}
\newcommand{\C}{\mathbb{C}}
\newcommand{\F}{\mathbb{F}}
\newcommand{\N}{\mathbb{N}}
\newcommand{\lnorm}{\vartriangleleft}
\newcommand{\rnorm}{\vartriangleright}
\newcommand{\id}{\text{id}}
\newcommand{\dd}[1]{\mathrm{d}{#1}}
\newcommand{\p}[1]{\left( #1 \right)}
\newcommand{\parder}[2]{\frac{\partial #1}{\partial #2}}
\newcommand{\legendre}[2]{\left(\frac{#1}{#2}\right)}
\makeatletter
\renewcommand\part[1]{
\ifnum\pdfstrcmp{\varline}{1}=0
    \vspace{.10in}\textbf{\\#1)}
  \else
    \textbf{#1)}
  \fi\renewcommand{\varline}{1}}
\makeatother
\makeatletter
\newcommand{\tpmod}[1]{{\@displayfalse\pmod{#1}}}
\makeatother
\renewcommand{\restriction}{\mathord{\upharpoonright}}\author{Konstantin Miagkov} 
\title{Homework 4: Invariants and Geometric Constructions}
\begin{document} 
%\setstretch{1}
\maketitle
\section{Homework}
\begin{prb}
100 coins of different sizes are lined up in a row from smallest to biggest, and it is allowed to switch two coins if there is exactly one coins between them. Is it possible to put all the coins in the reverse order from biggest to smallest using such operations?
\end{prb}

\begin{prb}
Consider two rays out of point $T$ forming an angle $\alpha$, and two points $A, B$ inside the angle. Construct a circle which passes through $A$ and $B$ and intersects the two rays at points $X, Y$ such that $TX = TY$. You may assume that such a circle exists for a given configuration of $A, B, T$ and $\alpha$.
\end{prb}

\section{Reading}

\begin{sol}[H3.1]
Consider the parity of the sum of all numbers on the board. When we replace two numbers with their difference, the numbers $x$ and $y$ are replaced by $x-y$. The contribution to the sum of all numbers changes from $x+y$ to $x-y$, so it changes by $2y$. This is an even number, which means that the parity of the sum of all numbers is indeed an invariant. Let us compute this invariant at the initial and final states. At the final state, we just have 0, so the parity is even. Initially we have $$1+2+\ldots+1998 = \frac{1998\cdot 1999}{2} = 999\cdot 1999$$ which is odd. One can also see that the sum is odd by noticing that there are exactly $999$ odd numbers in the sum. Regardless, the invariant values at the initial and final states are different, which means it is impossible to achieve the final state from the initial one.
\end{sol}

\begin{sol}[L3.4]
Pick three arbitrary points $A, B, C$ on the circle. If $O$ is the center, we know that $AO = BO = CO$. Thus $O$ lies both on the perpendicular bisector of $AB$ and on the perpendicular bisector of $BC$. We can construct those perpendicular bisectors, and then intersect them to find $O$.
\end{sol}



\end{document}