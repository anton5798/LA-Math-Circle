\documentclass[a4paper,12pt]{article} 
\usepackage{amsmath}
\usepackage[retainorgcmds]{IEEEtrantools}
\usepackage{graphicx}
\usepackage{amsfonts}
\usepackage{amsthm}
\usepackage{centernot}
\usepackage{setspace}
\usepackage{mathtools}
\usepackage{amssymb}
\usepackage{bm}
\usepackage[mathscr]{euscript}
\usepackage{pictexwd,dcpic}
\usepackage{tikz}
\usepackage{tikz-cd}
\usepackage[margin=1in]{geometry}
\usepackage{breqn}
\newtheoremstyle{perfect}% name
  {}%         Space above, empty = `usual value'
  {}%         Space below
  {}% Body font
  {}%         Indent amount (empty = no indent, \parindent = para indent)
  {\bfseries}% Thm head font
  {.}%        Punctuation after thm head
  {\newline}% Space after thm head: \newline = linebreak
  {}%         Thm head spec
\theoremstyle{perfect}
\newtheorem{lem}{Lemma}
\newtheorem{thm}{Theorem}
\newtheorem{dfn}{Definition}
\newtheorem{exm}{Example}
\newtheorem{prop}{Proposition}
\newtheorem{crl}{Corollary}
\newtheorem{rem}{Reminder}
\newtheorem{prb}{Problem}
\newtheorem{sol}{Solution}
\newtheorem{exe}{Exercise}
\makeatletter
\newenvironment{cprb}[1]
  {\count@\c@prb
   \global\c@prb#1
    \global\advance\c@prb\m@ne
   \prb}
  {\endprb
   \global\c@prb\count@}
\makeatother
\makeatletter
\newenvironment{cexe}[1]
  {\count@\c@exe
   \global\c@exe#1
    \global\advance\c@exe\m@ne
   \exe}
  {\endexe
   \global\c@exe\count@}
\makeatother
\newcommand{\varline}{0}
\newcommand{\gen}[1]{\left< #1 \right>}
\newcommand{\ngen}[1]{\gen{\gen{#1}}}
\newcommand{\ov}[1]{\,\overline{#1}}
\DeclareMathOperator{\tor}{Tor}
\DeclareMathOperator{\aut}{Aut}
\DeclareMathOperator{\inn}{Inn}
\DeclareMathOperator{\im}{im}
\DeclareMathOperator{\imm}{Im}
\DeclareMathOperator{\ad}{ad}
\DeclareMathOperator{\Ad}{Ad}
\DeclareMathOperator{\Sp}{Sp}
\DeclareMathOperator{\SO}{SO}
\DeclareMathOperator{\SL}{SL}
\DeclareMathOperator{\SU}{SU}
\DeclareMathOperator{\GL}{GL}
\DeclareMathOperator{\PGL}{PGL}
\DeclareMathOperator{\re}{Re}
\DeclareMathOperator{\Hom}{Hom}
\DeclareMathOperator{\sym}{Sym}
\DeclareMathOperator{\ind}{Ind}
\DeclareMathOperator{\res}{Res}
\DeclareMathOperator{\sgn}{sgn}
\DeclareMathOperator{\End}{End}
\DeclareMathOperator{\colim}{colim}
\DeclareMathOperator{\coker}{coker}
\DeclareMathOperator{\Tr}{Tr}
\DeclareMathOperator{\intr}{int}
\DeclareMathOperator{\extr}{ext}
\DeclareMathOperator{\chr}{char}
\DeclareMathOperator{\supp}{supp}
\DeclareMathOperator{\hol}{Hol}
\DeclareMathOperator{\spec}{Spec}
\renewcommand{\Re}{\re}
\renewcommand{\Im}{\imm}
\newcommand{\eps}{\varepsilon}
\newcommand{\Mor}{\text{Mor}}
\newcommand{\cir}[1]{\mathrlap{\bigcirc}{\;#1}}
\newcommand{\Z}{\mathbb{Z}}
\newcommand{\Q}{\mathbb{Q}}
\newcommand{\R}{\mathbb{R}}
\newcommand{\C}{\mathbb{C}}
\newcommand{\F}{\mathbb{F}}
\newcommand{\N}{\mathbb{N}}
\newcommand{\lnorm}{\vartriangleleft}
\newcommand{\rnorm}{\vartriangleright}
\newcommand{\id}{\text{id}}
\newcommand{\dd}[1]{\mathrm{d}{#1}}
\newcommand{\p}[1]{\left( #1 \right)}
\newcommand{\parder}[2]{\frac{\partial #1}{\partial #2}}
\newcommand{\legendre}[2]{\left(\frac{#1}{#2}\right)}
\makeatletter
\renewcommand\part[1]{
\ifnum\pdfstrcmp{\varline}{1}=0
    \vspace{.10in}\textbf{\\#1)}
  \else
    \textbf{#1)}
  \fi\renewcommand{\varline}{1}}
\makeatother
\makeatletter
\newcommand{\tpmod}[1]{{\@displayfalse\pmod{#1}}}
\makeatother
\renewcommand{\restriction}{\mathord{\upharpoonright}}\author{Konstantin Miagkov} 
\title{Homework 5: Invariants and Geometric Constructions}
\begin{document} 
%\setstretch{1}
\maketitle
\section{Homework}
\begin{prb}
Numbers $1, 2, \ldots, 20$ are written at the board. Every operation erases two numbers $a,b$ are replaces them by $a+b-1$. What are all the numbers that could be left after the operation is applied 19 times?
\end{prb}

\begin{prb}
Given a $\triangle ABC$, let $A_1, B_1, C_1$ be the midpoints of sides $BC, AC$ and $AB$ respectively. Show that the center of the circumcircle of $\triangle ABC$ is the same as the intersection of the altitudes of $\triangle A_1B_1C_1$. You may use the result of problem L5.5.
\end{prb}

\section{Reading}

\begin{sol}[H4.1]
Let us number the positions of all coins from 1 to 100. In order to reverse the order of all the coins, the coin at position 1 has to end up at position 100. Now consider every single switching operation, and suppose we are switching coins at positions $n$ and $n+2$. Then coin at position $n$ moves to position $n+2$, coin at position $n+2$ moves to position $n$, and the rest of the coins do not move. Then note that the parity of the position of each coin does not change. But then coins 1 could never move from position 1 to position 100, contradiction.
\end{sol}

\begin{sol}[L4.3]
Solution 1:\\

\noindent Construct the midpoint $M$ of $AB$, and draw a circle $\omega_1$ with center $M$ and radius $MA$. Then this is a circle with diameter $AB$. Now draw a circle with center $A$ and radius $CD$, and consider an intersection point $P$ of this circle with $\omega_1$. Since $P$ lies on the circle with diameter $AB$, we know that $\angle APB = 90^\circ$. We also have $PA = BC$, which means that $\triangle PAB$ is our desired triangle.\\
Solution 2:\\

\noindent Draw an arbitrary line $\ell$ through an arbitrary point $E$, and construct a line $m$ through $E$ perpendicular to $\ell$. Draw a circle with center $E$ and radius $BC$, and let $D$ be the intersection of $m$ and this circle. Now draw a circle through $D$ with radius $AB$, and let $T$ be the intersection of this circle with $\ell$. Then $ED = BC$, $DT = AB$ and $\angle DET = 90^\circ$, which means the triangle $\triangle EDT$ is the right triangle we wanted to construct.
\end{sol}



\end{document}