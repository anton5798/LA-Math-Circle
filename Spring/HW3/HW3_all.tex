\documentclass[a4paper,12pt]{article} 
\usepackage{amsmath}
\usepackage[retainorgcmds]{IEEEtrantools}
\usepackage{graphicx}
\usepackage{amsfonts}
\usepackage{amsthm}
\usepackage{centernot}
\usepackage{setspace}
\usepackage{mathtools}
\usepackage{amssymb}
\usepackage{bm}
\usepackage[mathscr]{euscript}
\usepackage{pictexwd,dcpic}
\usepackage{tikz}
\usepackage{tikz-cd}
\usepackage[margin=1in]{geometry}
\usepackage{breqn}
\newtheoremstyle{perfect}% name
  {}%         Space above, empty = `usual value'
  {}%         Space below
  {}% Body font
  {}%         Indent amount (empty = no indent, \parindent = para indent)
  {\bfseries}% Thm head font
  {.}%        Punctuation after thm head
  {\newline}% Space after thm head: \newline = linebreak
  {}%         Thm head spec
\theoremstyle{perfect}
\newtheorem{lem}{Lemma}
\newtheorem{thm}{Theorem}
\newtheorem{dfn}{Definition}
\newtheorem{exm}{Example}
\newtheorem{prop}{Proposition}
\newtheorem{crl}{Corollary}
\newtheorem{rem}{Reminder}
\newtheorem{prb}{Problem}
\newtheorem{sol}{Solution}
\newtheorem{exe}{Exercise}
\makeatletter
\newenvironment{cprb}[1]
  {\count@\c@prb
   \global\c@prb#1
    \global\advance\c@prb\m@ne
   \prb}
  {\endprb
   \global\c@prb\count@}
\makeatother
\makeatletter
\newenvironment{cexe}[1]
  {\count@\c@exe
   \global\c@exe#1
    \global\advance\c@exe\m@ne
   \exe}
  {\endexe
   \global\c@exe\count@}
\makeatother
\newcommand{\varline}{0}
\newcommand{\gen}[1]{\left< #1 \right>}
\newcommand{\ngen}[1]{\gen{\gen{#1}}}
\newcommand{\ov}[1]{\,\overline{#1}}
\DeclareMathOperator{\tor}{Tor}
\DeclareMathOperator{\aut}{Aut}
\DeclareMathOperator{\inn}{Inn}
\DeclareMathOperator{\im}{im}
\DeclareMathOperator{\imm}{Im}
\DeclareMathOperator{\ad}{ad}
\DeclareMathOperator{\Ad}{Ad}
\DeclareMathOperator{\Sp}{Sp}
\DeclareMathOperator{\SO}{SO}
\DeclareMathOperator{\SL}{SL}
\DeclareMathOperator{\SU}{SU}
\DeclareMathOperator{\GL}{GL}
\DeclareMathOperator{\PGL}{PGL}
\DeclareMathOperator{\re}{Re}
\DeclareMathOperator{\Hom}{Hom}
\DeclareMathOperator{\sym}{Sym}
\DeclareMathOperator{\ind}{Ind}
\DeclareMathOperator{\res}{Res}
\DeclareMathOperator{\sgn}{sgn}
\DeclareMathOperator{\End}{End}
\DeclareMathOperator{\colim}{colim}
\DeclareMathOperator{\coker}{coker}
\DeclareMathOperator{\Tr}{Tr}
\DeclareMathOperator{\intr}{int}
\DeclareMathOperator{\extr}{ext}
\DeclareMathOperator{\chr}{char}
\DeclareMathOperator{\supp}{supp}
\DeclareMathOperator{\hol}{Hol}
\DeclareMathOperator{\spec}{Spec}
\renewcommand{\Re}{\re}
\renewcommand{\Im}{\imm}
\newcommand{\eps}{\varepsilon}
\newcommand{\Mor}{\text{Mor}}
\newcommand{\cir}[1]{\mathrlap{\bigcirc}{\;#1}}
\newcommand{\Z}{\mathbb{Z}}
\newcommand{\Q}{\mathbb{Q}}
\newcommand{\R}{\mathbb{R}}
\newcommand{\C}{\mathbb{C}}
\newcommand{\F}{\mathbb{F}}
\newcommand{\N}{\mathbb{N}}
\newcommand{\lnorm}{\vartriangleleft}
\newcommand{\rnorm}{\vartriangleright}
\newcommand{\id}{\text{id}}
\newcommand{\dd}[1]{\mathrm{d}{#1}}
\newcommand{\p}[1]{\left( #1 \right)}
\newcommand{\parder}[2]{\frac{\partial #1}{\partial #2}}
\newcommand{\legendre}[2]{\left(\frac{#1}{#2}\right)}
\makeatletter
\renewcommand\part[1]{
\ifnum\pdfstrcmp{\varline}{1}=0
    \vspace{.10in}\textbf{\\#1)}
  \else
    \textbf{#1)}
  \fi\renewcommand{\varline}{1}}
\makeatother
\makeatletter
\newcommand{\tpmod}[1]{{\@displayfalse\pmod{#1}}}
\makeatother
\renewcommand{\restriction}{\mathord{\upharpoonright}}\author{Konstantin Miagkov} 
\title{Homework 3: Combinations and Pascal's Triangle}
\begin{document} 
%\setstretch{1}
\maketitle

\section{Homework}

\begin{prb}
How many ways are there to arrange 2 green, 2 red and 2 blue balls in a row so that not two balls of the same color are adjacent to each other?
\end{prb}

\begin{prb}
Let $T$ be a point inside a circle, and $A, B, C, D$ be points on the circle such that lines $AB$ and $CD$ intersect at $T$. Show that $AT\cdot TB = CT \cdot TD$.
\end{prb}

\section{Reading}

\begin{sol}[L2.3]
\textit{If the ratios of three binomial coefficients are} $${{n+1} \choose {m+1}} : {{n+1} \choose {m}} : {{n+1} \choose {m-1}} = 5 : 5 : 3$$ \textit{find $n,m$}.
Since $${{n+1} \choose {m+1}} : {{n+1} \choose {m}} : {{n+1} \choose {m-1}} = 5 : 5 : 3$$ we know that $${{n+1} \choose {m+1}} = {{n+1} \choose {m}}$$
Writing it out we get 
$$\frac{(n+1)!}{(m+1)!(n-m)!} = \frac{(n+1)!}{(m)!(n+1-m)!}$$
$$(m)!(n+1-m)! = (m+1)!(n-m)!$$
Cancelling out $m!(n-m)!$ we get $n+1-m = m+1$ or $n = 2m$.
Now we use 
$${{n+1} \choose {m}} : {{n+1} \choose {m-1}} = 5 : 3$$ 
to do the same factorial calculations:
$$\frac{3(n+1)!}{m!(n+1-m)!} = \frac{5(n+1)!}{(m-1)!(n+2-m)!}$$
$$3(m-1)!(n+2-m)! = 5m!(n+1-m)!$$
Cancelling out $(m-1)!(n+1-m)!$ we get $3(n+2-m) = 5m$. Plugging in $n= 2m$ we have $$3m + 6 = 5m$$
$$m = 3, n = 6$$
and we are done.
\end{sol}

\begin{sol}[H2.1]
If we want to descend to the $n$-th row of Pascal's triangle, we have to make $n$ total steps, since every step takes us one row down. At each step we can choose to go left or right, so the number of ways is $2^n$.
\end{sol}





\end{document}